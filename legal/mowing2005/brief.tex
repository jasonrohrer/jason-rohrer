\documentclass[12pt]{article}
\usepackage{fullpage,times}

\begin{document}


\begin{center}
POTSDAM VILLAGE COURT

STATE OF NEW YORK
\end{center}

\vspace{0.5in}

\noindent
\begin{tabular}{ll}

\begin{tabular}{l|}
\hline
\\
\parbox{4in}{
  \begin{tabular}{ll}
  THE PEOPLE OF THE STATE OF NEW YORK.&\\
  \end{tabular}
\\
\\
   -- versus --
\\
\\
  \begin{tabular}{ll}
  JASON ROHRER,&\\                                 
  &Defendant.\\
  \end{tabular}
}\\ \\
\hline
\end{tabular}
& Case No: 05080018
\end{tabular}


\vspace{2in}


\begin{center}
{\bf 
MEMORANDUM OF LAW IN SUPPORT OF\\
REQUEST FOR DECLARATORY JUDGMENT FINDING\\
ORDINANCE TO BE UNCONSTITUTIONAL
} 

\end{center}

\vspace{2in}


\begin{flushright}

\begin{tabular}{l}
Jason Rohrer\\
93 Elm Street\\
Potsdam, NY 13676\\
(315) 265-0585\\
\\
{\em Pro Se} Defendant
\end{tabular}

\end{flushright}


\newpage




\tableofcontents

\newpage

\section{Natural Landscaping Background}

In the United States, the natural landscaping movement has been growing in popularity over the last 30 years.
The movement seeks to landscape residential, commercial, and public properties in ways that are in harmony with natural environments and habitats.

Traditional landscaping generally seeks to dominate nature and sculpt the land into an unnatural, manicured state.
Traditional landscaping is still carried out on the vast majority of residential properties in the U.S.---its hallmarks are expanses of closely-cut lawn, mulched planting beds, and sculpted shrubbery (sometimes trimmed into geometric shapes such as cubes and cones).     

Natural landscaping was conceived partly in reaction to the problems associated with traditional landscaping.
These problems include:
\begin{itemize}
\item loss of wildlife and native plant habitat;  
\item lack of vegetative diversity;  
\item air pollution gases released by freshly-cut grass;  
\item heavy use of water;  
\item reliance on chemical fertilizers; and
\item dependence on noisy, petroleum-powered equipment for regular maintenance.
\end{itemize}
The natural landscaping movement was also motivated by an aesthetic sentiment:  nature is beautiful in its diversity, balance, and sustainability.
Instead of viewing properties as domains under complete control, natural landscapers view them as environments that can be interacted with and learned from. 

Many forces stand in opposition to the natural landscaping movement.
First, since the movement flies in the face of traditional landscaping preferences, social and cultural forces oppose it.
Those who attempt natural landscaping will surely ``stand out'' in their neighborhood as unusual, since they will be surrounded by traditional landscapes.

A more serious threat to natural landscapers is the force of government.
Most municipalities in the U.S. still have so-called ``mowing ordinances'' or ``weed control laws'' on the books that effectively prohibit natural landscaping.
These ordinances are usually strictly enforced, even against those who are cultivating natural landscapes.

On the other hand, government bodies have started to recognize the legitimacy, as well as the fundamental importance, of natural landscaping efforts.
For example, the U.S. Environmental Protection Agency has funded the publication of {\em  A Source Book on Natural Landscaping for Public Officials}, which can be found online at:
\begin{center}
http://www.epa.gov/glnpo/greenacres/toolkit/index.html
\end{center}

Though no case involving natural landscaping or mowing has ever been heard in a federal court or a in New York state appeals court with authority over the Village of Potsdam, cases heard before other state courts are certainly relevant.
The most noteworthy case is {\em City of New Berlin v.\ Donald Hagar}, No.\ 33582 (Wisconsin Circuit Court, Waukesha City, April 21, 1976).  
The defendant's situation in {\it Hagar} closely parallels the situation described below.
The Court struck down pertinent portions of New Berlin's mowing ordinance as unconstitutional.

The decision from {\it Hagar} is presented in Appendix D of Bret Rappaport's 1993 article on weed laws that appeared in {\em The John Marshall Law Review}.
A copy of the article, including Appendix D, is attached to this memorandum.


\section{Statement of Facts}

\subsection{Our plan of natural landscaping}
On June 14, 2004, my wife Lauren Serafin and I (Jason Rohrer) purchased the residential property at 93 Elm Street in the Village of Potsdam, New York.
The house was delivered to us with a traditional landscape:  an expansive and frequently-mowed lawn dotted with various manicured planting beds.

We immediately began a natural landscaping process on the property.
To explain our actions to neighbors and others passing by, we displayed signs on the north and east street frontage lines of the property that read:  ``Natural Meadow Restoration in Progress:  Please do not trample or disturb the grasses and other vegetation.''
A copy of the sign is attached to this memorandum.
We refrained from mowing the property and encouraged a diverse population of plant growth in areas that were formerly mowed lawns and manicured beds. 

A month into this process, we could already observe the effects of our efforts.
What was once a lawn was now speckled with a variety of plant species, including tall grasses and many different types of wildflowers.
Some of these plants were several feet tall and certainly would not have been able to develop in a mowed environment.
We also observed a variety of wildlife inhabiting our property, including insects such as butterflies and dragonflies, as well as birds.

\subsection{The Potsdam Village law}

In July of 2004, I obtained a copy of the property maintenance code.
At that time, Section 145-6.B(2) of the Potsdam Village Code read:
\begin{quote}
Grass shall be cut on improved property at least once every two (2) weeks and on vacant parcels of land at least once every three (3) weeks from the first day of May to the first day of October of each year.  
This provision shall not apply to land used for farm purposes.
\end{quote}
I inquired at the Code Enforcement Office about an exception for natural landscaping and was told that flowers are allowed, but lawns must be mowed.
We decided to continue pursuit of our natural landscaping project, reasoning that we no longer had a lawn but instead had a wildflower meadow, so the mowing law did not apply.

\subsection{First enforcement of the law}

On August 16, 2005, I received a call from the Code Enforcement Office.
The officer asked me if I was planning to mow my lawn any time soon.
I told him that we had a wildflower meadow, not a lawn, and we were trying to pursue natural landscaping.
He said he would ask his supervisor if an exception was possible.

After that call, also on August 16, 2005, I decided to document the current state of our yard.
I counted 40 different types of flowers, many of them wild, growing in a naturalized state where the lawn used to be.
I was in my front yard taking pictures of these flowers when the code enforcement officer, John Hill, arrived.
He took pictures of our property and served us a notice that our yard was in violation of Section 145-6.
The only motivation given in the notice for the mowing ordinance was alluded to by the closing sentence:  ``Thank you for your cooperation in keeping our village beautiful.''
A copy of the notice is attached to this memorandum.

The notice indicated that if we did not mow our yard in five days, the Village would have it mowed and charge us for this service, along with a 25\% surcharge.

I explained to Officer Hill that we were trying to pursue natural landscaping and that mowing would effectively destroy the landscape that we were cultivating.
I pointed out the flowers that were growing in the middle of our yard that would be cut down by a lawn mower.
He said he was only doing his job in enforcing the law and indicated that he could not make an exception for our particular kind of landscape.
He also indicated that he was only concerned about the front and side yards of our property---he did not plan on enforcing the ordinance in our back yard.

I pointed out a property to Officer Hill that was visible from my driveway:  the wide strip of land that extends to the street between 101 and 103 Elm Street.
This particular property, I pointed out, was unmowed and yet was clearly visible from the street.
Why, I asked, were the code enforcers not targeting that property, which was clearly in violation?
Officer Hill indicated that he was not sure why, and that they would probably not target the property with a notice unless his office received a complaint about it.
I told him that I would like to register a complaint about it.
He seemed to ignore me, and I told him that I better see him out in front of that particular property next week taking pictures.
He became visibly angry at this point and told me not to tell him how to do his job.

He suggested that the matter concerning our yard at 93 Elm Street should be settled in court.
He then left our property.

Later that day, also on August 16, 2004, I spoke with Village Administrator Michael Weil by telephone.
I told him about the unequal enforcement of the law and about the nearby properties that were allowed to remain unmowed.
He told me that, as they look into the issue further, they may end up enforcing the ordinance against those other properties also.

\subsection{Response of the Potsdam Village Board}

On August 17, 2004, I attended a meeting of the Village Board and spoke at the public comment session.
Officer Hill was present at the meeting.
I explained natural landscaping to the board and delivered some printed materials about natural landscaping, including an example mowing ordinance that had been amended to make an exception for natural landscapes.

My comments were largely dismissed by the mayor and the board.
For example, Mayor Ruth Garner said, ``I value my head too much to even consider it sir.  Most people like their lawns neat and are trying to dispel wildlife.''
One trustee, Helen Brouwer, did take my suggestions seriously and promised to read the materials I had delivered.
This interaction was described in the August 18, 2004 edition of the {\em Potsdam Massena Courier Observer} in the article ``Wildlife Preserve On Elm?'' by Robert Snow.
A copy of the article is attached to this memorandum.
 

\subsection{Initial compliance}

We contemplated going to court at that time, but decided against this option because we had a poor understanding of constitutional law and felt unprepared to present our defense.
To comply with the law while still pursuing a limited---and certainly unsatisfying---form of natural landscaping, we reasoned that we would need to remove all grass in the front and side yards.
This reasoning stemmed from the fact that the mowing ordinance applied only to grass and not to other types of plants.

In an attempt to remove the grass, we mechanically tilled these meadow areas under.
This left a barren landscape of tilled earth that was fully compliant with the Potsdam Village Code---we notified the Village in writing of our compliance, as requested by the notice.

In the fall of 2004, we spread leaf mulch in the front and side yards in a further attempt to kill whatever grass had survived the tilling operation.

\subsection{An amended law}

On May 12, 2005, the Village amended Section 145-6.B(2) to read:
\begin{quote}
Grass shall not exceed ten inches (10'') in height.  
This provision shall not apply to land used for farm purposes.
\end{quote}

In light of hearsay about this new ``ten inch rule,'' we decided to plant a low-growing vegetation in the front and side yards that would not exceed ten inches.
We selected white clover for its low growth habits, hospitability to wildlife, and minimal environmental impact.

Though the clover became successfully established in a few parts of our front yard, it failed to become fully established.
It did not become established all in the side yard.
Natural grass and other plant growth competed with the newly-planted clover and pushed it out.
Our front and side yards returned to a naturalized state containing a diversity of vegetation types, including grasses that were over ten inches tall.

\subsection{Second enforcement of the law}

On July 28, 2005, we received another notice that our yard was in violation of Section 145-6.
A copy of the notice is attached to this memorandum.\footnote{
Note that the second notice is the same as the first notice, but it is missing the bottom half (absent are the final two paragraphs and the officers' signatures).
Officer Hill explained that Code Enforcement Office had a stack of pre-printed notices left over when the law changed, so they decided to use up the notices (instead of wasting them) by simply cutting them in half to omit reference to the out-dated two-week requirement.
This footnote is relevant because it explains why the second notice does not mention ``keeping our village beautiful.''} 

On July 29, 2005, I approached Officer Hill and told him that I would like to contest the notice.
He said he was not sure what the procedure was and that I should talk to the village administrator.
He told me that he would hold off on sending a contractor to mow our yard as long as there was litigation pending.
He explained that tall grass was a fire hazard, mentioning the possibility of a passing driver flicking a cigarette onto our property.
He indicated that the rationale for the mowing ordinance was to prevent fires.



I approached the administrator, Michael Weil, and described our situation, informing him that I was contesting the notice that I received.
Since our first notice in 2004, I had ample opportunity to educate myself about constitutional law and the legality of natural landscaping.

I informed Mr.\ Weil that the mowing ordinance, as it stood, was unconstitutional.
I said that landscaping was a form of expression, and an ordinance that effectively forbids natural landscaping violated my First Amendment rights.
I told him that any law restricting First Amendment rights must further a compelling state interest, and that an arbitrary aesthetic standard was not a compelling interest.
He disagreed with me that the aesthetic standard was not a compelling interest, saying that neighbors have the right to a pleasant view of the properties around them.
He said that property values were also a compelling state interest, and said that natural landscaping would lower the values of neighboring properties.
He indicated that the majority of people wanted to see mowed lawns in the Village.

I handed him a printed copy of a Wisconsin court decision, {\em City of New Berlin v.\ Donald Hagar}, No.\ 33582 (Wisconsin Circuit Court, Waukesha City, April 21, 1976), that found a mowing ordinance to be unconstitutional when it was applied against a natural landscape.
Mr.\ Weil indicated that he did not believe the decision in {\em Hagar} had any bearing on the constitutionality of Potsdam's mowing ordinance.
He said that if he did enough research, he was sure he could find example cases where the courts have sided with municipalities that are trying to force natural landscapers to mow.

I told him that the Village mowing ordinance was being enforced in an unconstitutional manner in violation of the equal protection clause.
I pointed out that there are unmowed properties visible from my front driveway that touch Elm Street.
I explained that these properties were also unmowed in 2004 and that I complained about them then.
He said that it may turn out that these properties will need to be mowed.

Mr.\ Weil said he was not sure what the procedure was for contesting the notice, and that he would have to consult a lawyer to figure out what to do.
He told me that he would contact me with further information. 

On August 3, 2005, after the initial five day period had expired, I still had not heard back from Mr.\ Weil or Officer Hill about my contest.
Fearing that our natural landscape would be destroyed against our will and without due process, I tried to contact both Mr.\ Weil and Officer Hill by phone, leaving them each a message.

Officer Hill returned my call and told me that he thought a court appearance ticket would be the most appropriate course of action.
I agreed with him.
On August 4, 2005, he visited our house at 93 Elm Street and delivered appearance ticket number 100346 to me in person.

\subsection{Survey of nearby properties that are unmowed}
On August 3, 2005, I conducted a brief survey of Village properties near 93 Elm Street that have unmowed street frontage.
I documented this survey with 22 date-stamped, digital photographs.
A video tape showing all of these photographs has been submitted with this memorandum.

The results of this survey are:
\renewcommand{\labelenumi}{(\Alph{enumi})}
\begin{enumerate}
\item the property between 101 and 103 Elm Street has grass over fifteen inches (15'') tall within six feet of the street (photos 2744 to 2750);
\item the property just beyond 102 Elm Street, up to the corner of Elm and Wellings Drive, has waist-high grasses approximately 20 feet from the street, as well as grasses that exceed ten inches in height between the ditch and the street(photos 2751 to 2756 and 2764 to 2768);
\item the property on the corner of Ridgewood Lane and Fairlawn Drive (across from the house at 10 Ridgewood and next to the house at 4 Ridgewood) has waist-high grasses that touch both streets and even hang over the road in some places (photos 2759 and 2761); and
\item the property just south of the house at 14 Fairlawn Drive has waist-high grass that touches the street---this particular meadow area continues for hundreds of feet to the south on Fairlawn (photos 2757, 2758, and 2760).
\end{enumerate}
A map showing the locations of the unmowed properties is attached to this brief.
All properties listed above are within the Village limits and less than 0.5 miles from our property on 93 Elm Street.
Two of the properties, (A) and (B), are visible from our property at 93 Elm Street. 

All of the properties described in this survey are on unimproved lots that are immediately adjacent to, and in full view of, improved properties that are traditionally landscaped.
Some of the adjacent improved properties are fastidiously manicured in the traditional manner.
For example, the house on 4 Ridgewood Lane has shrubbery trimmed into spiral formations (photo 2762).


Note that the Village mowing ordinance makes no distinction between improved and unimproved properties.
All grass must be trimmed to less than ten inches in height.
The previous incarnation of the ordinance did make a distinction between improved and unimproved lots, requiring unimproved properties to be mowed less frequently (once every three weeks instead of once every two weeks).  

The properties adjacent to these illegal, unmowed areas are by no means low-value, at least not when compared to other properties in the Village.
For example, here are the assessment values of four properties in the Wellings/Ridgewood/Fairlawn allotment that have direct views of unmowed areas:
\begin{center}
\begin{tabular}{|l|r|}
\hline
{\bf Property} & {\bf Assessment}\\
\hline
\hline
4 Ridgewood & \$105,000\\
\hline
6 Ridgewood & \$150,000\\
\hline
10 Ridgewood & \$137,700\\
\hline
4 Fairlawn & \$150,000\\
\hline
\end{tabular}
\end{center}
At the time of the 2000 census, the median housing value in the Potsdam Village was \$73,900.

None of the properties described in this survey are being used for farm purposes.


\subsection{Surveys of our natural landscape}
On August 5, 2005, I conducted a brief survey of the wildflowers growing on our property.
I found a variety of native and non-native species, including 
\begin{itemize}
\item columbine ({\em Aquilegia canadensis}), 
\item black-eyed susan ({\em Rudbeckia hirta}),
\item narrowleaf plantain ({\em Plantago lanceolata}),
\item daisy fleabane ({\em Erigeron philadelphicus}),
\item goldenrod ({\em Solidago canadensis}),
\item white campion ({\em Silene alba}),
\item field daisy ({\em Chrysanthemum leucanthemum}),
\item queen anne's lace ({\em Daucus carota}),
\item ladysthumb ({\em Polygonum persicaria}),
\item common thistle ({\em Cirsium vulgare}),
\item hop clover ({\em Trifolium aureum}),
\item evening primrose ({\em Oenothera biennis}),
\item white melilot ({\em Melilotus alba}), and
\item great mullein ({\em Verbascum thapsus}).   
\end{itemize}  
Some of these wildflower plants were more than six feet tall.

On August 7, 2005, I conducted a brief survey of the tall grasses growing on our property.
I found six distinct varieties, each with a uniquely-structured seed head, including
\begin{itemize}
\item green foxtail ({\em Setaria viridis}),
\item timothy ({\em Phleum pratense}), and
\item orchard grass ({\em Dactylis glomerata}).
\end{itemize}
Some of these grasses were more than two feet tall.

\subsection{Habitat certification}

Our natural landscape has been certified as a wildlife habitat by the National Wildlife Federation.
Our Habitat certification number is 55082.
A copy of the certificate is attached to this memorandum. 

The Federation supplied us with a metal certification sign that we have posted in front of our landscape. 


\section{Question Before the Court}

The question in this case is whether the Village government has the power to enforce conformity in residents' choice of landscaping. 

In arguing below that the answer to this question is ``no,'' I show that such legislation is unconstitutional both on its face and in the way that it is being enforced by Village in this particular case.


\section{Argument}

Section 145-6.B(2) of Potsdam Village Code, which outlaws grass that exceeds ten inches in height, suppresses expression in a way that violates the free speech clauses of the Unites States and New York State Constitutions.
This ordinance is enforced in a selective and arbitrary manner that violates the equal protection clauses of the Unites States and New York State Constitutions.

These arguments are expanded below.


\subsection{Freedom of expression}

\subsubsection{Landscaping is protected expression}
Landscaping is undoubtably a form of expression.
When one chooses between tulips or daisies; maple or pine trees; and shrubs or boulders, one expresses a personal aesthetic sentiment.
Is a specific message conveyed by landscaping expression?
Furthermore, is landscaping an attempt to communicate?

In {\em Hurley v.\ Irish-American Gay, Lesbian \& Bisexual Group of Boston} (94-749), 515 U.S. 557 (1995), the Court stated its opinion on the necessity of specific messages to the First Amendment protection for particular forms of expression:
\begin{quote}
[A] narrow, succinctly articulable message is not a condition of constitutional protection, which if confined to expressions conveying a 'particularized message,' cf. {\em Spence v.\ Washington}, 418 U.S. 405, 411 (1974) (per curiam), would never reach the unquestionably shielded painting of Jackson Pollock, music of Arnold Sch\"onberg, or Jabberwocky verse of Lewis Carroll.
\end{quote}
In {\em Hurley}, the Court also pointed out that ``the Constitution looks beyond written or spoken words as mediums of expression.'' 
Though landscaping, like a Jackson Pollock painting, may not contain a particular message, it is a protected form of expression.

Looking beyond specific spoken or written messages, however, we can see a symbolic form of communication underlying the choices of a landscaper.
Traditional landscapers, perhaps unknowingly, express their dominance over nature with their manicured lawns and shrubs.
But this expression of dominance is not being restricted by the Village, so speculation about it is irrelevant to this case.
On the other hand, the symbolic expression possible with natural landscaping is relevant.

As natural landscapers, we intentionally try to communicate through our landscaping choices:  we wish to express our respect for nature and the environment.
Natural landscaping makes a symbolic statement against the ecological damage that accompanies traditional landscaping techniques.
By demonstrating that environmentally-friendly landscaping can also be beautiful, natural landscapers are advocating social (and in this case, political) change.
  
In {\it West Virginia State Board of Education v.\ Barnette}, 319 U.S. 624 (1943), the Court recognized the importance of symbols in expression, noting that 
``[s]ymbolism is a primitive but effective way of communicating ideas.''
When neighbors or others passing by see our natural landscape, they see a symbolic rejection of the traditional landscaping form.

This symbolic message can be made more clear when signs are posted to explain the natural landscape.
As mentioned in the Statement of Facts above, we did post such signs around our property to explain our landscaping choices.
When the symbolism of the landscape is accompanied by a sign about meadow restoration or habitat certification, the expression of a respect for nature is clear.
Our landscape did in fact convey a particularized message.

Thus, natural landscaping generally, and particularly as we practice it on our property at 93 Elm Street, is a form of expression that enjoys full First Amendment protection.

\subsubsection{The test for permissible suppression}

The Court has permitted the government to suppress certain forms of protected expression in particular circumstances.
For example, content-neutral time, place, and manner restrictions for public expression have often been upheld.  

In {\it United States v.\ O'Brien}, 391 U.S. 367, 377 (1968), the Court set the following four-part deferential standard:
\begin{quote}
[A] government regulation is sufficiently justified if it is within the constitutional power of Government; if it furthers an important or substantial governmental interest; if the governmental interest is unrelated to the suppression of free expression; and if the incidental restriction on alleged First Amendment freedom is no greater than is essential to the furtherance of that government interest.
\end{quote}

Regarding the mowing ordinance in question, the governmental interest is an aesthetic one:  the original violation notice read, ``Thank you for your cooperation in keeping our village beautiful.''
Furthermore, the ordinance targets grass, a particular type of landscape content.
The ``ten-inch rule'' applies only to grass and not to flowers, vegetable plants, shrubs, or trees.
Thus, the Village is effectively permitting particular kinds of aesthetic content (for example, tall flowers) while forbidding others (tall grasses).
Those landscapers who wish to express themselves with tall flowers are allowed to do so, while those that wish to express themselves with tall grasses have their expression suppressed.

It follows that the government interest is directly related to the suppression of free expression.
The law was probably not formulated with suppression in mind, but instead with some seemingly noble sentiment about maintaining visual character.
However, underlying this sentiment is a desire to promote one form of landscaping expression (that which the majority deems beautiful) while suppressing another (that which the majority deems ugly).
Thus, the ordinance fails the third requirement set in {\em O'Brien}.

The Village has extended its aesthetic motivation to include the supposed protection of property values.
On its face, this motivation is unrelated to the suppression of free expression.
However, even if the Court wants to apply deferential scrutiny here, the {\em O'Brien} standard requires that a law actually {\em furthers} an important interest, not just that it aims to protect such an interest.
The Village has not provided any evidence to support its claim that natural landscaping reduces the values of neighboring properties.
If the assessments from the Wellings/Ridgewood/Fairlawn allotment are any indication, perhaps just the opposite is actually true:  we could imagine that those properties which enjoy the beauty and tranquility of nearby natural areas may actually be more valuable than properties surrounded by traditional landscaping.
Thus, the Village's property-value motivation fails the second requirement laid by {\em O'Brien}.

The Village has also mentioned fire prevention as an interest furthered by the mowing ordinance, though it has not indicated that this is its primary motivation (the violation notices, after all, read ``Thank you for your cooperation in keeping our village beautiful,'' not ``Thank you for your cooperation in keeping grass fires at bay'').
The fire prevention interest is founded on the idea that tall grass is a serious fire hazard.

In fact, tall grass is not a fire hazard, as argued by Rappaport's article in {\em The John Marshall Law Review}, which is attached to this memorandum.
Because grass burns quickly and produces very little heat, grass fires are not dangerous.
Such a fire would not produce enough heat to ignite nearby structures.
Furthermore, grass fires do not produce sufficient wind-born embers to ignite structures or vegetation on neighboring properties.
Thus, because the mowing ordinance does not actually further the interest of fire prevention, this motivation also fails the second requirement laid by {\em O'Brien}.




Since the ordinance, as it applies to natural landscaping, fails to satisfy the {\em O'Brien} standard, the Court must apply strict scrutiny.
Only a substantially-compelling government interest can warrant the suppression of particular kinds of expression, and the Village government has demonstrated no such compelling interest to justify the suppression of natural landscaping.

Indeed, natural landscaping, with its positive environmental impact, furthers more public interests than traditional landscaping.
Neighbors to traditional landscapes are subjected to weekly assaults against their senses from the noises and odors generated by lawn mowers.
Neighbors to natural landscapes are not assaulted in this way.
Furthermore, the environment of the public as a whole, both in- and outside the Village, is assaulted by the activities of traditional landscapers:  air pollution gases are released by lawn-mowing equipment and by the cut grass itself.
Pollution directly from cut grass is discussed in the 2002 article ``Air pollution and the smell of cut grass,'' by Kirstine, Galbally, and Hooper.
This article was published at the 16th International Clean Air Conference in Christchurch, New Zealand.
A copy of the article is attached to this memorandum.

The only possible offense of a natural landscape against a neighbor is a visual one, but this is an arbitrary matter of taste---people who hate roses may feel unduly assaulted by views of their neighbors' rose gardens.
As the Court pointed out in  {\em Cohen v.\ California}, 403 U.S. 15 (1971), ``one man's vulgarity is another's lyric.''
The Court went on to observe that ``it is largely because governmental officials cannot make principled distinctions in this area that the Constitution leaves matters of taste and style so largely to the individual.''

Those who are visually offended by our natural landscape, like those courthouse-attendees in {\em Cohen}, can ``effectively avoid further bombardment of their sensibilities simply by averting their eyes.''

The Court expanded on this idea in  {\em Erznoznik v.\ City of Jacksonville}, 422 U.S. 205 (1975), observing: 
\begin{quote}
The plain, if at times disquieting, truth is that in our pluralistic society, constantly proliferating new and ingenious forms of expression, ``we are inescapably captive audiences for many purposes.'' {\em Rowan v.\ Post Office Dept.}, supra, at 736. Much that we encounter offends our esthetic, if not our political and moral, sensibilities. Nevertheless, the Constitution does not permit government to decide which types of otherwise protected speech are sufficiently offensive to require protection for the unwilling listener or viewer.
\end{quote}
Here, the Village is attempting to legislate aesthetic taste in landscaping to protect those who are offended by tall grasses---such legislation clearly exceeds its power. 

If it was just the enforcement against natural landscapes that was in question, the Court might subject the mowing ordinance to a narrow interpretation and declare that it does not apply to natural landscapes.
However, the ten-inch ordinance is unconstitutional on its face, and not just as applied here, because it is overbroad.
The Court must consider a whole spectrum of possible landscaping expressions that might be possible with grass that exceeds ten inches.
One type of expression, to be sure, is a natural landscape, but others may exist.
To uphold the mowing ordinance, or even a narrow interpretation of it, could potentially have a chilling effect on the landscaping expression of others.


\subsubsection{Other possible government interests}
\label{section:other_interests}
Though the Village has offered no other motivations for its mowing ordinance, several additional motivations have been cited by other municipalities to justify similar ordinances.
When scrutinized, these motivations have not been shown to be firmly based on facts.

These motivations, each accompanied by fact-based scrutiny, are:
\begin{itemize}
\item {\bf Tall grass produces wind-borne pollen that aggravates human allergies.}  In fact, wind-borne pollen can travel hundreds of miles.  
There is no evidence that a nearby natural landscape will substantially contribute to the allergies of people in the vicinity.
\item {\bf Tall grass harbors rodents.}  In fact, rodents are attracted to areas that have adequate food supplies, such as garbage dumps or grain silos.
Tall grass alone will not attract rodents.
\item {\bf Tall grass harbors mosquitos.}  In fact, mosquitos need access to standing water for ten days to breed.
Mosquitos cannot breed in tall grass without standing water.
\end{itemize}
These facts about tall grass were taken from Rappaport's article in {\em The John Marshall Law Review}.
As noted earlier, a copy of the article is attached to this memorandum.

\subsection{Two related cases}

In the sections above, we have examined a broad spectrum of case law related to freedom of expression. 
Several types of non-speech expression were found to be immune to government suppression or interference.
In summary, people have the right to:
\begin{itemize}
\item display modified versions of the U.S. flag ({\em Spence}),
\item abstain from saluting the flag ({\em Barnette}),
\item wear jackets bearing the word ``fuck'' into a public courthouse ({\em Cohen}), and
\item display nudity on drive-in screens that are in full view of a public street ({\em Erznoznik}).
\end{itemize}
As a free society, we certainly have the right to express ourselves in many ways.
But one question still has not been answered explicitly by the Court:  do citizens in this free society have the right to landscape their private properties however they choose?
Phrased differently, can municipalities enforce arbitrary aesthetic standards on private properties?
None of the cases discussed so far have addressed this particular issue.
Instead, they dealt with speech that was being suppressed for reasons other than purely aesthetic ones (for example, maintaining national pride or protecting children from profanity).

One case is closely related in terms of subject matter. 
In {\em City of Ladue v.\ Gilleo}, 512 U.S. 43 (1994), the question before the Court was whether a city could prohibit the display of certain types of signs on residential properties.
In particular, Gilleo wanted to display a sign advocating ``peace in the Gulf'' on her property
Though Ladue's ordinances allowed ``for sale'' signs to be posted, it forbid the kind of sign Gilleo sought to display.
The motivations given by the city of Ladue were that such signs
\begin{quote} 
would create ugliness, visual blight and clutter, tarnish the natural beauty of the landscape as well as the residential and commercial architecture, impair property values, substantially impinge upon the privacy and special ambience of the community, and may cause safety and traffic hazards to motorists, pedestrians, and children.
\end{quote}
The Court overturned Ladue's ordinance, noting that:
\begin{quote}
A special respect for individual liberty in the home has long been part of our culture and our law, see, e.g., Payton v.\ New York, 445 U.S. 573, 596-597, and nn.\ 44-45 (1980);  that principle has special resonance when the government seeks to constrain a person's ability to speak there.   See Spence v.\ Washington, 418 U.S. 405, 406 , 409, 411 (1974)
\end{quote}
The Court also found the selective suppression of certain types signs to be impermissible, with arguments highlighting the fact that permissible regulations for expression must be content-neutral.

Though the Court has limited the government's regulation of expression on private property, it has given government some leeway when in dealing with public property.
In {\em City Council of Los Angeles v.\ Taxpayers for Vincent}, 466 U.S. 789 (1984), the Court upheld a a city ordinance that prohibited the posting of signs on public property such as utility poles.
The city's desire to prevent ``visual clutter'' on public property was found to be sufficiently compelling.
The Court explicitly dealt with the issue of public versus private property as follows:
\begin{quote}
The validity of the City's esthetic interest in the elimination of signs on public property is not compromised by failing to extend the ban to private property.  The private citizen's interest in controlling the use of his own property justifies the disparate treatment...
\end{quote}
The {\em Vincent} decision also emphasized the content-neutral requirement for expression regulation, observing that the ``general principle that the First Amendment forbids the government to regulate speech in ways that favor some viewpoints or ideas at the expense of others'' was not being violated by the Los Angeles ordinance.
Unlike the Ladue ordinance, the Los Angeles ordinance applied to all types of signs, so it passed constitutional muster,  

In both {\em Gilleo} and {\em Vincent}, the question before the Court was about how far the government could go in furthering its aesthetic interest.
In {\em Gilleo}, the Court found that government reached the limit when it tried to restrict particular kinds of expression, while allowing others, on private property.

With the Potsdam mowing ordinance, we have a strikingly-similar restriction.
Particular forms of landscaping are being restricted (those, such as natural landscaping, with grass that exceeds 10 inches) while others are being permitted (such as traditionally-mowed landscaping).
Furthermore, the restriction tramples individuals' rights to express themselves on their own properties.  
Finally, the restriction primarily seeks to further the Village's aesthetic interest.



\subsection{Equal protection}

The enforcement of the Village mowing ordinance is carried out in a selective and arbitrary manner.
There are unmowed properties throughout the Village that are both visible from the street and adjacent to improved properties.
Four of these properties are surveyed in the Statement of Facts above.
The Village mowing ordinance is not being actively enforced against these properties, even when complaints are lodged about this selective enforcement. 

The unmowed properties in question are all unimproved lots, whereas our property at 93 Elm Street is improved.
Thus, the apparent distinction made by the Village is between improved and unimproved properties.
As mentioned in the Statement of Facts, the law makes no distinction between improved and unimproved lots.
Looking at the older version of the mowing law, we can conclude that the intention is for unimproved lots to be mowed (the old law required mowing once every three weeks), and that the new law simply switches to an across-the-board standard that is easier to enforce.

There is no rational basis for this differential enforcement that could be related to any goal of the Village mowing ordinance.

Regarding the motivations that the Village has actually proposed (aesthetics, property values, and fire hazards), tall grass on unimproved lots would certainly have the same visual character, the same potential to reduce neighboring property values, and the same potential to catch fire as tall grass on improved lots.  

Turning to the other possible motivations discussed in Section \ref{section:other_interests}, even if these motivations were based firmly on facts, there is no evidence that the government interest would only be relevant on improved properties.
In other words, tall grass on unimproved lots could potentially produce pollen, harbor rodents, and harbor mosquitos just as the tall grass on an improved property could.



%\subsection{Procedural Due Process}

%\subsection{Substantive Due Process}


\section{Request for Relief}

I request that the Court issue a declaratory judgment finding that the mowing ordinance in Section 145-6.B(2) of the Potsdam Village Code is unconstitutional.
\begin{flushright}
\begin{tabular}{l}
Respectfully submitted,\\
\\
\\
\hline
\\
Jason Rohrer, {\em pro se}\\
93 Elm Street\\
Potsdam, NY 13676
\end{tabular}
\end{flushright}
Date:\underline{\hspace{2in}}

\end{document}
