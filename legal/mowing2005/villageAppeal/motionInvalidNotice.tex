\documentclass[12pt]{article}
\usepackage{fullpage,times,setspace}

\begin{document}


\begin{center}
ST. LAWRENCE COUNTY COURT

STATE OF NEW YORK
\end{center}

\vspace{0.3in}

\noindent
\begin{tabular}{ll}

\begin{tabular}{l|}
\hline
\\
\parbox{4in}{
  \begin{tabular}{ll}
  THE PEOPLE OF THE STATE OF NEW YORK.&\\
  \end{tabular}
\\
\\
   -- versus --
\\
\\
  \begin{tabular}{ll}
  JASON ROHRER,&\\                                 
  &Defendant.\\
  \end{tabular}
}\\ \\
\hline
\end{tabular}
& \begin{tabular}{l}
Case No.: 17890\\
Return Date: November 13, 2006\\
Return Time: 9:30 am 
\end{tabular}
\end{tabular}

\vspace{0.3in}


\doublespacing

\begin{center}
{\bf 
NOTICE OF MOTION
} 

\end{center}

\vspace{0.1in}




\noindent PLEASE TAKE NOTICE that upon my Affidavit, sworn to on October 23, 2006, and upon all of the papers and evidence included in the return filed by the Potsdam Village Court, a motion will be made at the above date and time for an Order granting the following relief:

\begin{enumerate}
\item Finding the People's Notice of Appeal to be fatally defective, on the grounds that it inaccurately describes the judgment being appealed from as a ``dismissal'' when it was in fact an acquittal; and
\item Dismissing the People's Appeal, on the grounds that the Appeal was not properly taken due to the defective Notice, and on the grounds that the New York State Criminal Procedure Law does not authorize the People to appeal acquittals.
\end{enumerate}

\noindent PLEASE TAKE FURTHER NOTICE that this motion is made on submission and no oral argument is requested.

\singlespacing

\begin{flushright}
\begin{tabular}{l}
Respectfully submitted,\\
\\
\\
\hline
\\
Jason Rohrer, {\em pro se}\\
93 Elm Street\\
Potsdam, NY 13676
\end{tabular}
\end{flushright}
%Date:\underline{\hspace{2in}}
Date: October 23, 2006

\end{document}
