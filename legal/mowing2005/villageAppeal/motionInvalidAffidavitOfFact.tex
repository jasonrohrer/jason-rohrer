\documentclass[12pt]{article}
\usepackage{fullpage,times,setspace}

\begin{document}


\begin{center}
ST. LAWRENCE COUNTY COURT

STATE OF NEW YORK
\end{center}

\vspace{0.5in}

\noindent
\begin{tabular}{ll}

\begin{tabular}{l|}
\hline
\\
\parbox{4in}{
  \begin{tabular}{ll}
  THE PEOPLE OF THE STATE OF NEW YORK.&\\
  \end{tabular}
\\
\\
   -- versus --
\\
\\
  \begin{tabular}{ll}
  JASON ROHRER,&\\                                 
  &Defendant.\\
  \end{tabular}
}\\ \\
\hline
\end{tabular}
& \begin{tabular}{l}
Case No.: 17890\\
\end{tabular}
\end{tabular}

\vspace{0.5in}


\doublespacing

\begin{center}
{\bf AFFIDAVIT OF FACTS}
\end{center}

\vspace{0.3in}

\noindent I, Jason Rohrer, being duly sworn, depose and say:

\begin{enumerate}

\item The original trial in the above-entitled matter was held on December 15, 2005 in the Potsdam Village Court.

\item \label{item:first_trial_feature} The trial lasted more than two hours.

\item Sworn witness testimony was received from Village Code Enforcement Officer John Hill and from me, the Defendant.  

\item Expert witness testimony was received from Richard Grover, a professional landscape architect.

\item Several exhibits were accepted into evidence.

\item The Village Court reserved its decision to receive summation papers from both parties.

\item \label{item:last_trial_feature}On June 12, 2006, the Village Court issued a verdict of ``not guilty'' in a six-page written decision.

\item On July 12, 2006, the People timely filed a Notice of Appeal.
 

\item The People's Notice uses the following wording (bold added):
\begin{quote}
``PLEASE TAKE NOTICE, that the undersigned hereby appeals to the New York State County Court, St.\ Lawrence County from the {\bf dismissal} of the charge entered in the Potsdam Village Court, County of St. Lawrence, in the above-entitled matter on the 12th day of June, 2006, and from each and every part thereof on the basis that the Judgment is contrary to the law and facts and on each and every other ground permissible pursuant to the Criminal Procedure Law of the State of New York.''
\end{quote}

\item The decision rendered by the Potsdam Village Court was not a dismissal, but rather an acquittal, as supported by the facts in paragraphs \ref{item:first_trial_feature} through \ref{item:last_trial_feature}.

\item No dismissals were issued by the Village Court at any time before or during the trial.

\item A dismissal, by definition, is a termination of prosecution proceedings before a trial is complete and a verdict has been reached.

\item The People's Notice does not accurately describe the Judgment being appealed from.

\item The People's Notice states that an appeal is permissible pursuant to the Criminal Procedure Law.
If the Judgment was in fact a dismissal, such an appeal would be permissible pursuant to that Law, but the Judgment was in fact an acquittal.

\item The New York State Criminal Procedure Law, Article 450, section 20, enumerates the rulings that can be appealed by the People, including various sentences, orders of dismissal, orders reducing counts, orders setting aside verdicts, orders vacating judgments, et cetera.
Section 40 of that same Article sets additional requirements for appeal of dismissals by the People.

\item Nowhere does Criminal Procedure Law, Article 450, authorize the People to appeal acquittals.

\item Concerning appeals by the government as they relate to double jeopardy jurisprudence and the Criminal Procedure Law, the distinction between a dismissal and an acquittal is so great that the inaccurate wording in the People's Notice must be viewed as a fundamental and fatal defect.

\end{enumerate}

\noindent {\bf WHEREFORE}, I respectfully request that the People's Notice of Appeal be found defective and that the Appeal be dismissed.


\singlespacing

\begin{flushright}
\begin{tabular}{l}
\\
\\
\hline
\\
Jason Rohrer, {\em pro se}\\
93 Elm Street\\
Potsdam, NY 13676
\end{tabular}
\end{flushright}

\begin{tabular}{l}
\\
\\
\hline
\\
Notary Public\\
\\
Sworn To on October \underline{\hspace{0.25in}}, 2006
\end{tabular}

\end{document}
