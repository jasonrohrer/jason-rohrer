\documentclass[12pt]{article}
\usepackage{fullpage,times}

\begin{document}

\noindent Jason Rohrer

\noindent July 17, 2006


\begin{center}
{\Huge Quotes about Double Jeopardy}
\end{center}

\section{The U.S. Constitution}

{\bf Amendment V - Trial and Punishment, Compensation for Takings.}  (Ratified December 15, 1791.)

\begin{quote}
``No person shall be held to answer for a capital, or otherwise infamous crime, unless on a presentment or indictment of a Grand Jury, except in cases arising in the land or naval forces, or in the Militia, when in actual service in time of War or public danger; {\bf nor shall any person be subject for the same offense to be twice put in jeopardy of life or limb;} nor shall be compelled in any criminal case to be a witness against himself, nor be deprived of life, liberty, or property, without due process of law; nor shall private property be taken for public use, without just compensation.''
\end{quote}


\section{The U.S. Supreme Court}

\subsection{Acquittals and Double Jeopardy}


\paragraph{United States v. DiFrancesco, 449 U.S. 117, 132 (1980):}

\begin{quote}
``It is an acquittal that prevents retrial even if legal error was committed at the trial.''
\end{quote}


\paragraph{United States v. Scott, 437 U.S. 82 (1978):}

\begin{quote}
``As Kepner and Fong Foo illustrate, the law attaches particular significance to an acquittal. To permit a second trial after an acquittal, however mistaken the acquittal may have been, would present an unacceptably high risk that the Government, with its vastly superior resources, might wear down the defendant so that `even though innocent he may be found guilty.' Green, 355 U.S., at 188.''
\end{quote}



\paragraph{United States v. Martin Linen Supply Co., 430 U.S. 564 (1977):}

\begin{quote}
``Perhaps the most fundamental rule in the history of double jeopardy jurisprudence has been that `[a] verdict of acquittal . . . could not be reviewed, on error or otherwise, without putting [a defendant] twice in jeopardy, and thereby violating the Constitution.' United States v. Ball, 163 U.S. 662, 671 (1896)''
\end{quote}


\newpage

\paragraph{Green v. United States, 355 U.S. 184 (1957):}

\begin{quote}
``The underlying idea, one that is deeply ingrained in at least the Anglo-American system of jurisprudence, is that the State with all its resources and power should not be allowed to make repeated attempts to convict an individual for an alleged offense, thereby subjecting him to embarrassment, expense and ordeal and compelling him to live in a continuing state of anxiety and insecurity, as well as enhancing the possibility that even though innocent he may be found guilty.''
\end{quote}
The Court continues in Green:

\begin{quote}
``[I]t is one of the elemental principles of our criminal law that the Government cannot secure a new trial by means of an appeal even though an acquittal may appear to be erroneous.''
\end{quote}



\subsection{Bench trials (without juries) and Double Jeopardy}



\paragraph{United States v. Jenkins, 420 U.S. 358 (1975):}

\begin{quote}
``Since the Double Jeopardy Clause of the Fifth Amendment nowhere distinguishes between bench and jury trials, the principles given expression through that Clause apply to cases tried to a judge.''
\end{quote}
The Court continues in Jenkins:

\begin{quote}
``Here there was a judgment discharging the defendant, although we cannot say with assurance whether it was, or was not, a resolution of the factual issues against the Government. But it is enough for purposes of the Double Jeopardy Clause, and therefore for the determination of appealability under 18 U.S.C. 3731, that further proceedings of some sort, devoted to the resolution of factual issues going to the elements of the offense charged, would have been required upon reversal and remand. Even if the District Court were to receive no additional evidence, it would still be necessary for it to make supplemental findings. The trial, which could have resulted in a judgment of conviction, has long since terminated in respondent's favor. To subject him to any further such proceedings at this stage would violate the Double Jeopardy Clause...''
\end{quote}



\paragraph{Smalis v. Pennsylvania, 476 U.S. 140 (1986):}

\begin{quote}
``[W]hether the trial is to a jury or to the bench, subjecting the defendant to postacquittal factfinding proceedings going to guilt or innocence violates the Double Jeopardy Clause.''
\end{quote}



\end{document}