\documentclass[12pt]{article}
\usepackage{fullpage,times}

\begin{document}


\begin{center}
POTSDAM VILLAGE COURT

STATE OF NEW YORK
\end{center}

\vspace{0.5in}

\noindent
\begin{tabular}{ll}

\begin{tabular}{l|}
\hline
\\
\parbox{4in}{
  \begin{tabular}{ll}
  THE PEOPLE OF THE STATE OF NEW YORK.&\\
  \end{tabular}
\\
\\
   -- versus --
\\
\\
  \begin{tabular}{ll}
  JASON ROHRER,&\\                                 
  &Defendant.\\
  \end{tabular}
}\\ \\
\hline
\end{tabular}
& Case No: 05080018
\end{tabular}


\vspace{2in}


\begin{center}
{\bf 
DEFENSE SUMMATION
} 

\end{center}

\vspace{2in}


\begin{flushright}

\begin{tabular}{l}
Jason Rohrer\\
93 Elm Street\\
Potsdam, NY 13676\\
(315) 265-0585\\
\\
{\em Pro Se} Defendant
\end{tabular}

\end{flushright}


\newpage




\tableofcontents

\newpage

\section{Freedom of Expression}

My testimony, and the testimony of expert witness Richard Grover, clearly proved that landscaping is a form of expression.
The testimony further demonstrated that natural landscaping can express a particular message about the property owners' respect for the environment and desire to live in harmony with nature.
We posted signs around our property at 93 Elm Street to further clarify our message, and a copy of one of these signs was accepted into evidence.

My memorandum cites extensive case law to back the notion that landscaping is a form of expression that should enjoy full constitutional protection.
There are very strict limits on the kinds of restrictions governments can place on protected expression.

\subsection{Restrictions of place}
Testimony showed that Officer Hill did not try to enforce the mowing ordinance in our backyard, and the prosecution suggested that I could accept this partial-enforcement as a compromise. 
The prosecution's suggestion that we restrict our natural landscaping to our backyard is unacceptable on equal protection grounds:  our neighbors are certainly not compelled to restrict their traditional landscaping to their backyards.
In addition, I have nothing but the word of Officer Hill to assure me that he will not enforce the mowing ordinance in our backyard sometime in the future.
My testimony about the natural landscape on Circle Drive demonstrates that Officer Hill has indeed forced natural landscapers to mow all parts of their yard, both back and front, until no natural landscaping was left at all.

Furthermore, our ability to speak with our landscape would be drastically reduced if our desired landscape was restricted to our backyard, since very few people could see it.

For example, lawn signs are a form of speech that has been explicitly protected by the Court's ruling in {\em City of Ladue v.\ Gilleo}, 512 U.S. 43 (1994).
Imagine a Village ordinance that allowed lawn signs, but only in the backyard.
This could certainly be seen as a ``content-neutral time, place, or manner'' restriction, but one that contradicts the spirit of the First Amendment protection established by {\it Gilleo}, since the backyard is generally a place that very few passerby can see.

Expert testimony showed that traditional mowed lawns pose serious environmental hazards to the neighbors in the form of air pollution, noise pollution, and run-off.
It is interesting to note that these hazards to neighboring properties would clearly {\it not} be mitigated by restricting mowed lawns to backyards.
On the other hand, Village officials apparently feel that the ``hazards'' associated with our natural landscaping would be mitigated if the landscape was restricted to our backyard, where few people could see it.
This demonstrates that the primary ``hazard'' involved is an aesthetic one.  

\subsection{Motivating government interest}

My testimony further establishes that the primary Village interest in the mowing ordinance is an aesthetic one.
The original notice, as served to me in 2004, mentioned ``keeping our village beautiful.''
The notice offered no other motivation for the enforcement of the law.

Richard Grover's testimony highlighted the fact that many people would not understand a natural landscape when they looked at it.
Concerning his own property, it might look like he had simply neglected to mow, since people unfamiliar with the process would not notice the hand-selection of particular plant varieties that he was doing.
He also testified that, though natural landscaping is growing in popularity at a rapid rate, the vast majority of people in our region still prefer traditional, mowed lawns.
Thus, to the majority, a natural landscape may indeed seem aesthetically unappealing.

Officer Hill offered no other motivation for the mowing ordinance during his testimony.
However, my testimony showed that fire prevention was a potential Village interest.
During one of my interactions with Officer Hill in 2005, he indicated that tall grass was a fire hazard that could catch our house on fire.

Richard Grover testified about an actual grass fire next to a house that caused only minor, superficial damage in the form of melted siding.
His testimony refutes the notion that tall grass is a serious fire hazard.
My memorandum also deals with this issue, citing widely-accepted science about the burning characteristics of tall grass.
Also, there was no testimony indicating that tall grass is the only landscaping feature that could catch fire, nor was there testimony showing that traditional landscaping elements, such as shrubbery, are not fire hazards to nearby structures. 

My testimony also indicated that property values were a potential Village interest.
However, the Village offered no evidence or testimony to support the notion that natural landscaping would hurt the values of neighboring properties.
The testimony of Mr.\ Grover refuted this notion:  a natural landscape would be an ``asset'' to the neighborhood.

Thus, the only government interest furthered by the mowing ordinance is an aesthetic one.

\subsection{Failing the {\it O'Brien} test}

My memorandum covers the four-part deferential standard set by the Court in {\it United States v.\ O'Brien}, 391 U.S. 367, 377 (1968).
According to that standard, the Court should grant the government deference to regulate activities in ways that may limit expression only if such regulation furthers an important state interest.
However, the {\it O'Brien} test also stipulates that the state interest in question cannot be related to the suppression of expression.

Though aesthetic distinctions are subtle, subjective, and generally beyond the constitutional scope of government, we might still call the maintenance of aesthetic standards in the Village an important interest.
However, aesthetics are intimately connected to expression, and the two cannot be separated, at least when dealing with intentional human activities (though a view of a mountain range can certainly be aesthetically pleasing without having anything to do with expression).

It is impossible to regulate aesthetics without simultaneously regulating expression.
Any aesthetic regulation that furthers no other government interest automatically fails the {\it O'Brien} test and cannot be granted deferential scrutiny.
Thus, any aesthetic regulation must be subjected to strict scrutiny by the Court. 

In this case, since the only interest furthered by the mowing ordinance is an aesthetic one, the Court must subject this application of the ordinance to strict scrutiny.

Regulations only survive strict scrutiny if they further a compelling government interest.
However, the Court has already given its opinion on the power of government to regulate aesthetics.
As cited in my memorandum, the Court made the following observation in {\em Erznoznik v.\ City of Jacksonville}, 422 U.S. 205 (1975): 
\begin{quote}
The plain, if at times disquieting, truth is that in our pluralistic society, constantly proliferating new and ingenious forms of expression, ``we are inescapably captive audiences for many purposes.'' [citation omitted]  Much that we encounter offends our esthetic, if not our political and moral, sensibilities. Nevertheless, the Constitution does not permit government to decide which types of otherwise protected speech are sufficiently offensive to require protection for the unwilling listener or viewer.
\end{quote}
In {\em Cohen v.\ California}, 403 U.S. 15 (1971), the court observed that ``it is largely because governmental officials cannot make principled distinctions in this area that the Constitution leaves matters of taste and style so largely to the individual.''
Thus, aesthetics alone are not a compelling government interest, as they are beyond the reach of acceptable government regulation in the first place.

We can further connect the Court's {\it Erznoznik} observation to the testimony that we heard.
Richard Grover testified that natural landscaping is a relatively new form of expression that the majority of people are still unfamiliar with, though he also said that the field was rapidly growing.
He also testified that natural landscaping carried several environmental benefits.
Thus, expression through natural landscaping fits the mold of an ``ingenious'' and ``proliferating'' new form of expression.
According to the Court, this is the kind of expression that demands protection from government regulation. 


\section{Equal Protection}
\label{equal_protection}

The photographic evidence that I presented proved that there were other properties in the Village with grass in violation of the mowing ordinance.
All of these properties were on unimproved lots, but testimony about the law demonstrated that there is no exception made in the law for unimproved properties.
In fact, the wording of the 2004 ordinance as shown on the back of my 2004 notice (which was accepted into evidence) explicitly specified that unimproved lots needed to be mowed (once every 3 weeks).
The 2005 ordinance (the 10-inch rule), which was enacted, according to Officer Hill, because it was easier to enforce, applies the same 10-inch standard to all properties and makes no exception for unimproved lots.   
My testimony also showed that Village officials, including Officer Hill, received complaints from me about these other violations in both 2004 and 2005, but that the mowing ordinance was never enforced against these properties. 

%Hill testified that his regular rounds cover ``most of the streets in the Village.''
%Obviously, his rounds cover Elm Street (since our property is on Elm Street, and two of the unmowed properties (marked A and B in my photographic presentation) were on Elm Street.
%Thus, we would expect Officer Hill not only to respond to complaints about these properties, but to notice them himself.  


\subsection{Farm exception}
While cross-examining me, the prosecution implied that some of the properties in my photographic survey have been ``hayed off'' in the recent past.
The implication is that, since the law makes an exception for farm purposes, the properties would be exempt from the mowing ordinance 
Of course, no testimony offered by the prosecution showed that any of these properties were indeed hayed off.
The only person offering such information was Mr.\ Lekki himself, and he was certainly never sworn in as a witness.

My testimony refutes the notion that some of these properties were hayed off:  I live within line-of-sight from property A and B and never saw evidence of either being hayed off.
In addition, the photographs of property A show some kind of crab grass growing in a relatively shady area.
In August of 2005, this grass was slightly taller than 15 inches, but in no way resembled hay.

Concerning properties C and D, hay-making would be a highly unusual activity on small, vacant lots in the Wellings-Ridgewood-Fairlawn allotment. 
Also, neither property had anything resembling hay on it:  C contained a mix of grass and woody material, while D contained a large proportion of Queen Anne's lace.

The only property in my photographic survey that contained anything at all resembling hay was property A. 
However, along with the hay-like brown grass in the back, the street frontage of the property contained green grass and Queen Anne's lace exceeding 10 inches in height.
Thus, even if part of property A was exempt from the ordinance because it was being used for farm purposes, the street frontage of property A was still in violation. 

\subsection{Rational basis for differential enforcement}

Differential enforcement of a law is generally forbidden by the equal protection clauses in our state and national constitutions.
However, the government may write or enforce the law differently for different classes if there is a rational basis motivating the unequal treatment.
In this case, the two classes seem to be improved and unimproved properties, though Village officials never indicated that this division was their intent.
Thus, not only have the Village officials failed to provide a rational basis, they have failed to demonstrate that the unequal treatment was an intentional practice or an agreed-upon policy.
However, they have clearly ignored my complaints about unimproved properties, so we must assume that in practice, the mowing ordinance is not enforced against unimproved lots. 

Testimony during the trial indicated that it was reasonable to enforce a 10-inch mowing ordinance against traditional landscapes with mowed lawns.
Of course, there is nothing stopping someone from cultivating a mowed lawn on an unimproved property.
The following question could then arise:  what would happen if the owners of the lawn on this unimproved property began neglecting their mowing chores?
We would then have a mowed lawn, albeit on an unimproved lot, that was in reasonable violation of the 10-inch rule.

Thus, testimony has shown that even the apparent class division is not rational or internally-consistent. 

Perhaps the intended class division is really between lawns and meadows, and such a class division may have a rational basis.
None of the properties in my photographic survey would be mistaken for lawns.
Thus, there may still be an equal protection issue in this case, but one concerning of misclassification.
With this track of analysis, Village officials have simply mistaken our property as a lawn and enforced the ordinance as they would against a lawn.
The equal protection issue is that our property is not getting the same protection as the other meadow properties in the Village.
I expand this analysis further in Section \ref{lawns_vs_meadows} below.



\section{Lawmakers' Intentions}

From his testimony, it was clear that Officer Hill was unfamiliar with the practice of natural landscaping.
For example, Officer Hill recalled ticketing a non-traditional landscape on Harrison Street in Canton, but could not recall the landscaper giving any reasons for his landscaping practices.
Officer Hill also testified to seeing signs on our property in 2004, but he did not recall reading them.
The signs in question were copies of the ``Natural Meadow Restoration'' sign that I submitted into evidence.

Given that Officer Hill is charged by the Village Board to enforce the Village property maintenance code, it is clear that the Board itself is generally unfamiliar with the practice of natural landscaping.

\subsection{Lawns versus meadows}
\label{lawns_vs_meadows}

During the prosecution's cross-examination of Richard Grover, quite a bit of time was spent trying to define the situations in which a ``10-inch rule'' could reasonably be applied.
Mr.\ Grover agreed that such a rule might be reasonable for mowed lawns (though he found the idea of a lawn ordinance to be generally unreasonable), but he clearly stated that such an ordinance would be unreasonable to apply to a meadow landscape.

Though the ``10-inch rule'' on its face seems to apply to any and all grass, it is clear that the lawmakers had mowed lawns, not natural meadows, in mind when they drafted the law.
Thus, the law cannot reasonably be applied to a meadow landscape.

Mr.\ Grover testified that mowing a meadow landscape down to 10 inches would effectively destroy the meadow.
He further testified that a ``10-inch rule,'' if applied to all landscapes containing grass, would have a ``chilling effect'' on those who seek to cultivate a meadow landscape.

It is also clear from Officer Hill's testimony that the majority of mowing ordinance violations are on properties with traditional, mowed lawns.
For example, his testimony indicated that out of the 35 properties that were served with a notice in 2005, only one property had an intentional, non-traditional landscape (our property).
The other properties were apparently in violation of the 10-inch rule because of traditional lawns that had been temporarily neglected.
Thus, Officer Hill has very little experience in enforcing the mowing ordinance against natural landscapes, especially when compared to the amount of experience he has in dealing with traditional lawns that have been neglected.

Here we have a law that was intended to deal with one issue (neglected lawns) being applied occasionally to deal with a completely different issue (the intentional practices of natural landscapers).
The only feature that a neglected lawn shares with a natural landscape is grass in excess of 10 inches.
This, as it turns out, happens to be the feature picked out by the law as the hallmark of a neglected lawn.
The wording of the law forces Officer Hill to treat a neglected lawn and a natural landscape in the same way, though in practice the two require completely different treatment.

When questioned about the reasonableness of a 10-inch rule for traditional, mowed lawns, Richard Grover testified that a 10-inch lawn would look quite ``unruly.''
Just as 10 inches is absurdly long for a mowed lawn, 10 inches is absurdly short for a natural meadow.
According to Mr.\ Grover's testimony, it would cease to be a meadow if it was kept mowed to 10 inches.

%Equivalent would be an 85 MPH speed limit intended for cars (which would be quite unruly) being applied occasionally to jet airplanes.
%Since jet airplanes simply cannot stay airborne at such low speeds, they would relegated to the ground to become nothing but oddly-shaped cars.

%Likewise, a meadow trimmed regularly to 10 inches would be nothing but a very odd lawn.  

As mentioned in Section \ref{equal_protection} above, Village officials did not enforce the mowing ordinance against certain meadow areas in the Village, even after receiving complaints about them.
If we do not interpret this selective enforcement as a violation of equal protection, we must conclude that Village officials had no intention of applying the mowing ordinance to meadow areas.
This conclusion supports the notion that the lawmakers never intended the 10-inch rule to apply to meadows.

\subsection{Transitional landscapes}

Of course, it is impossible to intentionally go from a lawn to a meadow without some intermediate steps.
There were certainly times shortly after we began to cultivate a meadow that our property resembled a neglected lawn.
For example, two weeks after we started our natural landscaping, our former lawn looked identical to any other lawn in the Village that had gone two weeks without mowing, except for the fact that we had posted signs explaining our actions as intentional ``Natural Meadow Restoration''.

However, two months passed from the time we acquired our property in 2004 and the time Officer Hill issued the first notice on August 16, 2004.
At that point, our yard looked quite different from the average neglected lawn:
I testified that I pointed wildflowers out to Officer Hill at the time when he served me my 2004 notice.
The average neglected lawn certainly would not have wildflowers in it.
On the other hand, our yard still looked quite different from the average meadow, since the plants had not become fully established.

In 2004, we tilled our front and side yards under in an attempt to remove the grass.
Richard Grover mentioned that tilling areas under was one of several standard techniques natural landscapers could use when trying to control the type of vegetation that grows in a given area.
At that point in 2004, and anytime after that, our property looked very different from the average neglected lawn.

According to testimony, in 2005, some areas of our front and side yards did have tall grasses growing, but other areas were dominated by non-grasses, like sunflowers and clover.
The clover growth, according to testimony, resulted from seeds that we spread intentionally in our yard. 
Officer Hill testified that, in addition to tall grass, he observed ``wildflowers'' on our property in 2005.
At the time of our 2005 notice, our property looked nothing like a lawn, though it still did not look like a fully-established meadow.
  
To draw a comparison, according to my testimony, our backyard in 2005 was much more meadow-like, though it was still not a fully-established meadow, even after a year of being unmowed.
Thus, it is clear that meadow restoration is necessarily a gradual process:  no matter what technique is used, it is impossible to jump instantly from a lawn to a full-fledged meadow.

In this case, the mowing ordinance is being applied to a naturally-landscaped property that is intentionally in the transition stage from a lawn to a meadow.
It is clear that the lawmakers intent in the mowing ordinance was for it to apply to mowed lawns and not to these transitional landscapes.

%Going back to the speed-limit analogy (which is admittedly quaint, but an uncannily-good analogy), we could again imagine a speed limit of 85 MPH on the books for all vehicles, and that the limit was drafted with only cars in mind.
%We could further imagine that the government officials would not enforce the limit against jet airplanes that happened to be flying at 200 MPH or faster, but they would strictly enforce the 85 MPH limit for all vehicles that were still on the ground.
%Of course, since a jet airplane needs to travel faster than 85 MPH on the ground before it can even take off, the enforcement of the law would effectively prevent any jet airplanes that were not already airborne from ever becoming airborne. 

By effectively enforcing the mowing ordinance against any grassy area that is not yet a full-fledged meadow, the Village government prevents areas that are not yet meadows from ever becoming meadows.
On the other hand, the Village is permitting already-established meadows to remain unmowed.

\subsection{Substantive due process}
Though there may be some conceivable rational basis for applying a 10-inch rule to mowed lawns, there is no rational basis for applying it to natural meadow landscapes or natural landscapes that are in intentional transition toward becoming meadows.
Thus, the application of the rule in this case violates the due process clauses of the United States and New York constitutions.

\section{Visual Obstructions:  The Prosecution's Red Herring}

Some of the testimony drawn by the prosecution indicated that there may have been a visual obstruction for motorists caused by the landscaping on our property.

The prosecution has brought up the issue of an alleged visual obstruction at 93 Elm street to distract the Court from the core issues in this trial.
I am not being tried for a violation of the visual obstruction ordinance (for vegetation that exceeds 3 feet in height near an intersection), but instead with a violation of the mowing ordinance (for grass that exceeds 10 inches in height anywhere at all on our property).

If there was a visual obstruction on our property in September at the corner of Elm and Gilmore, it was obviously not due merely to a violation of the 10-inch rule, so the issue of a visual obstruction is not at all connected to my violation of the mowing ordinance.
If some feature of our landscape was a problem for motorists, the issue should have been handled via a separate notice of violation (preferably in September, shortly after a complaint was received), and not saved for more than 3 months to be shoe-horned by the prosecution into my mowing ordinance trial.

Especially if public safety is at stake, we would expect Officer Hill to act swiftly upon receiving such a complaint, though his testimony indicated that he neither informed us of a potential problem nor visited our property to investigate the problem himself.

I hope that the Court will not be distracted by the prosecution's ``surprise testimony'' tactics concerning visual obstructions, as we had no intention of endangering public safety with our landscaping.
If a potential visual obstruction was brought to our attention in September, we would have been willing to work with Officer Hill to mitigate the problem.

I will also point out to the Court that I was never arraigned or allowed to plead about visual obstruction charges, so this side-issue should be completely ignored by the Court when deciding my guilt or innocence.
The prosecution was trying to bring new and surprising charges of a distinct violation right in the middle of trial, completely abridging my right to due process. 
    

\section{Request}

The testimony, evidence, and case law show that the mowing ordinance is being enforced against our natural landscape in a manner that violates the free speech, equal protection, and due process clauses of the United States and New York constitutions.

I request that the Court return a verdict of ``not guilty'' on these grounds.

   


\begin{flushright}
\begin{tabular}{l}
Respectfully submitted,\\
\\
\\
\hline
\\
Jason Rohrer, {\em pro se}\\
93 Elm Street\\
Potsdam, NY 13676
\end{tabular}
\end{flushright}
Date:\underline{\hspace{2in}}

\end{document}
