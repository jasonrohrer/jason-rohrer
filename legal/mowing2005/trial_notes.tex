\documentclass[12pt]{article}
\usepackage{fullpage,times}


\begin{document}

\begin{center}
{\Huge Trial Notes}
\end{center}

{\bf Stand whenever speaking}


\section{Opening Statement}

Ask to reserve my statement until after the prosecution presents its case.


\section{Prosecution's Case}

\subsection{Prosecution's witnesses}
\begin{itemize}
\item The only admissable witness is Officer John Hill
\item The only admissible evidence is the law itself and photographs taken by Officer Hill.
\item Proescution failed to disclose any other witnesses to me before the trial or any other pieces of evidence.  

Object to any other witnesses or evidence (``Objection,  this witness was not disclosed to me before the trial'').
\end{itemize}

\subsection{I cross-examine prosecution's witnesses}
(I have the right to ask for a few minutes to think about my questions.)


\section{Defendant's Case}

\subsection{Opening statement}
\begin{enumerate}
\item Introduce self (representing myself)
\item Defending my constitutional right to cultivate a natural landscape at 93 Elm St.
\item landscape clearly violates section 145-6 (the ``mowing ordinance'')
\item I will testify about my natural landscape and about interactions with Village officials.
My testimony plus evidence will demonstrate two things:
        \begin{enumerate}
        \item my natural landscaping practices are a form of expression
        \item ordiance enforced in selective and arbitrary manner
        \end{enumerate}
\item One expert witness, Richard Grover.
Expert in landscape architecture, environmental issues, and community planning.
His expert testimony will  establish the legitimacy and environmental importance of natural landscaping.

\item Filing a memorandum of law

\item testimony and evidence, plus case law in memorandum, will show:
      \begin{enumerate}
      \item mowing ordinance violates free speech clauses (US and NY Const)
      \item enforcement of ordinacne violates equal protection clauses (US and NY Const)
      \end{enumerate}

\item Your honor---hope that after hearing the testimony, seeing the evidence, and reading the case law, that you will issue a declaratory judgement striking down the mowing ordinance as unconstitutional.
\end{enumerate}


\subsection{My witnesses}

How to admit evidence:
\begin{enumerate}
\item Mark exhibit
\item give it to prosecution to examine
\item authenticate exhibit myself (``these are the pictures I took'')
\item ask the judge to admit exhibit into evidence
\end{enumerate}


\subsubsection{My personal testimony}

\begin{enumerate}
\item reason for natural landscaping
\item summer of 2004
\item {\bf EXHIBIT A} sign hung in summer of 2004
\item interaction with John Hill (pointing out 101-103) and Michael Weil
\item {\bf EXHIBIT B} first notice
\item ``keeping village beautiful''
\item (don't talk about village board here)
\item initial compliance
\item summer 2005
\item {\bf EXHIBIT C} second notice
\item conversation with John Hill and Michael Weil (other unmowed properties, told him I pointed them out in 2004 already)
\item {\bf EXHIBIT D} (tape of) 22-photograph survey taken on August 3, 2005 of nearby properties that are not mowed 
\item {\bf EXHIBIT E} map that goes along with survey
\item {\bf EXHIBIT F} legend that goes along with survey
\item present 22-photograph survey
\item survey of wildflowers (read from memorandum)
\item {\bf EXHIBIT G} habitat certificate
\item sign provided by NWF posted on property
\item new england asters filled yard with purple this fall
\item sunflowers spontaneously grew in front yard in fall
\end{enumerate}

\subsubsection{Expert Witness Richard Grover}


\section{Closing Arguments}
\subsection{Prosecution's argument}

\subsection{My argument}

\paragraph{O'Brien quote}
In {\it United States v.\ O'Brien}, 391 U.S. 367, 377 (1968), the Court set the following four-part deferential standard:
\begin{quote}
[A] government regulation is sufficiently justified if it is within the constitutional power of Government; if it furthers an important or substantial governmental interest; if the governmental interest is unrelated to the suppression of free expression; and if the incidental restriction on alleged First Amendment freedom is no greater than is essential to the furtherance of that government interest.
\end{quote}


\paragraph{Cohen quotes}
As the Court pointed out in  {\em Cohen v.\ California}, 403 U.S. 15 (1971), ``one man's vulgarity is another's lyric.''
The Court went on to observe that ``it is largely because governmental officials cannot make principled distinctions in this area that the Constitution leaves matters of taste and style so largely to the individual.''

Those who are visually offended by our natural landscape, like those courthouse-attendees in {\em Cohen}, can ``effectively avoid further bombardment of their sensibilities simply by averting their eyes.'


\paragraph{Erznoznik quote} The Court expanded on this idea in  {\em Erznoznik v.\ City of Jacksonville}, 422 U.S. 205 (1975), observing: 
\begin{quote}
The plain, if at times disquieting, truth is that in our pluralistic society, constantly proliferating new and ingenious forms of expression, ``we are inescapably captive audiences for many purposes.''  Much that we encounter offends our esthetic, if not our political and moral, sensibilities. Nevertheless, the Constitution does not permit government to decide which types of otherwise protected speech are sufficiently offensive to require protection for the unwilling listener or viewer.
\end{quote}

\end{document}