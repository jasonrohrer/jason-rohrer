In the past, it was common for municipal governments to seek out big box retailers, offer incentives, and welcome them into town to put up stores.  More recently, many municipalities have dropped their ``open arms'' policies toward big boxes, because in spite of what these local governments once thought, big box retailers aren't very good for local economies.

%There used to be a fad for municipal governments to seek out big box  retailers and offer them most or all of what they wanted to come to a town and put up a store.  That fad is petering out because in spite of what the municipal governments thought, big box retailers aren't very good for local economies.

Big box stores {\em are} big.  As big as a warehouse or a factory.  Usually they're built on the outskirts of a city or town, where land is cheap.  They can't do this without the right zoning.  (See how municipal governments come into this?)

One thing big box stores do is kill downtowns.  With at least the promise of the lowest prices (not always true), they redirect consumer traffic away from the original downtown business core (the best place to zone for commercial development) out to their highway stores.  After that, patterns of highway traffic encourage sprawl development, with buildings and parking lots lining both sides of the highway in what used to be green space.  Many of these structures would be big enough to put a big strain on municipal services and the natural environment.  And after the downtown competition is eliminated, big box stores are perfectly free to raise those low prices if they want to.  And some sure do.

\vspace{6.5pt}

Big box stores don't add many jobs to local economies.  Most jobs are just transferred from closing stores in the original business district that the big box replaces.  Big boxes don't add anything to the tax base either, because the commercial properties they compete against decline in value and are taxed less.  Taxes on homes increase to make up the difference.

\vspace{6.5pt}

There's a thing called the multiplier effect.  Every time a dollar is spent, lent, or saved in a local economy, it's like a new dollar from outside.  At each transaction, local merchants spend their dollars locally in this way for advertising in local newspapers, for office supplies, for accounting services, repair work, and Internet services.  Big boxes don't help keep dollars in the local economy.  They ``outsource'' a lot of work, even for services like accounting, to out-of-state vendors and their own national distribution network.  Big boxes tend to put their money in local banks only long enough to wire it to corporate headquarters in another state.  As a percentage of profits, charitable giving by local merchants is also vastly more than what big box retailers donate.

% needed to trick last column into aligning
%\vspace{1pt}

\vspace{6.5pt}

To the extent that the bargains they offer are real (and local market-basket surveys have called that into question), big box retailers pay for them the low way.  They force their suppliers to cut {\em their} prices, so that the jobs manufacturing the products are themselves shipped overseas.  The jobs go to sweatshops in countries like China (and in China, some of this manufacturing work is forced labor by political prisoners).  Saying no to big boxes won't eliminate our trade deficit with China tomorrow, or make sweatshop goods disappear from our shelves the day after that, but it's a good place to start.  Big boxes are a big part of both those stories.

\vspace{7pt}

On the local scene, downtown Potsdam is coming back to life without the ``help'' of any big box stores in the vicinity, and there's a lot more we can do, working together as a community, for more jobs, better wages, and lower prices.  We don't need to endure the negative effects of big box stores, because we don't need big box stores, period.
