\documentclass[12pt]{letter}
\usepackage{fullpage,times}

\begin{document}

\begin{letter}{}

\address{93 Elm St.\\Potsdam, NY 13676}

\signature{Jason Rohrer}

%\newcommand{\recipientName}{Representative McHugh}
%\newcommand{\recipientName}{Senator Clinton}
\newcommand{\recipientName}{Senator Schumer}


\opening{\recipientName:}


I just finished reading {\em The FairTax Book} by Neal Boortz and Congressman John Linder.  
I think the FairTax is a great way to simplify our federal tax system without changing the total amount of revenue that our government brings in.


Our current tax system is too complicated.  
I know several people who are so intimidated by the tax regulations that they never file their federal returns, even when the IRS owes them a refund.  
These are not wealthy people, but instead people who are struggling just to get by.  


The FairTax simplifies our tax system dramatically, replacing tens of thousands of pages of tax code with 130 pages of legislation.  
This simplicity will eliminate the burden and hidden costs of tax compliance from our lives.


In essence, the Fair Tax Act of 2005 (H.R.\ 25 and S.\ 25) repeals all federal income taxes, payroll taxes (for both Social Security and Medicare) investment taxes, corporate taxes, estate taxes, and gift taxes.  
The Act replaces these taxes with a national sales tax on all goods and services at the retail level.
  


To ensure that this new taxation is progressive, the plan also specifies that a monthly tax rebate check should be sent to each and every household (both poor and rich).  
This rebate is an estimate of how much a household of a given size would pay in sales tax when buying the basic necessities of life for the month.  
Necessities, up to the poverty level, are tax-free.  


I just learned that two more representatives have joined the 45 other co-sponsors of the Fair Tax Act:  Spencer Bachus of Alabama and John Sullivan of Oklahoma.  
I hope you will demonstrate leadership in this area and sign on as a co-sponsor of this bill yourself.

Many people hear about the FairTax and say, ``Great idea, but we'll never get it passed.''  
My only response is this:  imagine, for a moment, if those pessimistic assessments were accurate---our state of affairs would be incredibly bleak.  
It is up to you, \recipientName, to ensure that the FairTax becomes a reality.
This bill deserves your full attention and effort.  

You must show us that the pessimists are wrong.



\closing{Sincerely,}

\end{letter}

\end{document}

