\documentclass[12pt]{article}
\usepackage{fullpage,times}

\pagestyle{empty}

\begin{document}

\noindent (A copy of this letter has also been sent by email)

\vspace{0.1in}

{\bf 
\noindent Letter to Editor:\\
Potsdam Spent \$105k on Wal-Mart}

\vspace{0.1in}


Freedom of Information Law (FOIL) requests have revealed that the Town of Potsdam has spent over \$105,000 for legal counsel and engineering work concerning the proposed Wal-Mart on Route 11.

Town officials promised that this project will not cost taxpayers a single penny.  On the surface, they seem to have kept their promise, since Wal-Mart is allowed to cover some of the Town's SEQR-related expenses.

However, section 617.13 of the SEQR law limits how much the Town can bill to Wal-Mart.  The limit is 0.5\% of the total project cost.  This project will cost \$8,400,000, so the limit is \$42,000.

At this point, the Town has passed the limit by more than \$63,000.  According to the law, the Town must cover this overage using its own funds.  However, Wal-Mart has apparently paid all the Town's SEQR-related expenses, long after the \$42,000 limit was surpassed.  

Have the Town and Wal-Mart been breaking New York State Law?

Furthermore, after the first FOIL request on July 7, 2005, strange things began to happen.

From February through June of 2005, the lawyers billed the Town's Wal-Mart-funded escrow account monthly, with net legal fees averaging over \$11,000 per month.  Suddenly, these bills stopped arriving, with the last invoice filed on June 13.

July, August, September, and October saw no more bills from the Town's lawyers.  Regular engineering bills continued to arrive, though these expenses averaged less than \$1200 per month from February through September.

Did the lawyers stop working for the Town?  Probably not.

Important events occurred during these months that required legal counsel.  For example, on September 13, the Planning Board met for FEIS approval with Special Counsel present.  Thus, the Town's lawyers worked in September, but they apparently never billed the Town for their work.

Did Special Counsel start working for free?

We might conclude that someone is hiding something.  Here we are, \$63,000 over the SEQR limit, with a lot more work in front of us.  Combine that with a promise not to use taxpayer money, and the Town has a strong incentive to hide the true cost of this project.

Why did the lawyers stop sending invoices after June?  Maybe because somebody started asking about the bills with a FOIL request in July.  Coincidence?

Mr. Demick, we deserve an explanation.

\vspace{0.5in}
\noindent Jason Rohrer\\
93 Elm St.\\
Potsdam, NY 13676\\
265-0585

\end{document}

