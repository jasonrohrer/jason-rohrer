\documentclass[12pt]{article}
\usepackage{fullpage,times}

%\pagestyle{empty}

\begin{document}

{\Huge

\begin{itemize}

\item \ [GP 1] Right off the bat, Judith is different from any recent artgame. [GP 6001] It gives you a full 360 degrees of freedom in a Wolfenstein-like, 2.5D, world.  [GP 6002]

\item \ [GP 6003] But unlike Wolf, there is a specific narrative being told as you move through the space. [Space once from GP 6003]

\item Little cut scenes strung here and there. Bits of dialog between you and other characters.
  
\item And yes, lots of bilboarded sprites.  Here's a billboarded flower garden. [GP 6004]

\item So, what characters?  A modern-day couple having a tryst on the grounds of a rented, historical estate. [GP 6005]

\item Emily and Jeff.  You control Jeff.

\item Emily gets lost, Jeff looks for her [GP 6006].

\item \ [Space once from GP 6006] But mixed in with this search are the interactions of another pair of characters, apparently from the distant past.

\item Judith and her husband, the historical owner of the estate.  You control Judith.

\item \ [GP 6007] She's exploring a hidden passageway in her home.  


\item \ [GP 6008] Discovering things about her husband that gradually terrify her.

\item In this passageway are a series of locked doors.  They mysteriously open, one by one, each day that Judith explores, revealing more and more disturbing secrets.

\item Folded into this, Jeff is looking for Emily.  The same series of locked doors opens one by one for him...


\item \ [GP 6009] There are even some false Yes/No branches (where the only answer that will advance the narrative is Yes).


\item Totally linear, unlocking the narrative one step at a time.

\item This seems like a disappointment.

\item Sort of the path of least resistance for narrative game design.

\item It sucks, right?

\item But then some weird things start to happen.  I'm going to show you a movie of one of them [GP 6010]

\item So, what's weird about that?  It wasn't actually a movie (show live game).
\item So why was Judith totally on rails there?  A First Person action cut scene?  A cop-out for a sequence that was too hard to program as interactive?

\item But consider Jeff's mirror sequence [GP 6011].  A totally interactive version of the same thing.

\item As the game goes on, Judith becomes less and less interactive.

\item \ [GP 6012] Finally, she is presented with a yes/no branch with only one choice.

\item Her final sequence is totally on rails as she walks to her death.

\item Jeff, meanwhile, is completely interactive throughout the game.

\item So:  what's going on here?

\item This game certainly has an interesting intertwining of past and present events in a shared space.  

\item Who's opening the doors for Judith, one by one?  Her husband, most likely.
\item But who's opening the doors for Jeff?  That question is the key to seeing the structure of this game as more than a cop-op.

\item This is a gated narrative, yes, but the mysteries of the gating itself are the keystone of the experience.

\item And why this interactivity contrast between the Judith and Jeff sequences?

\item This is a game ABOUT control itself.  Judith's interactivity declines as she succumbs to the diabolical trap set by her husband.  

\item You couldn't make the same game, with the same artistic effect, without the gating and the on-rails segments. 

\item So this is NOT just the standard game design cop-out.  This is an intentional, well-justified artistic choice.  And it works.  

\end{itemize}
}

\end{document}