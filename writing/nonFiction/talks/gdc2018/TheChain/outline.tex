\documentclass[12pt]{article}
\usepackage{fullpage,times,verbatim}

%\pagestyle{empty}


\begin{document}

\begin{center}
Don't Break the Chain

Maintaining Productivity on Your 19th Game 

Jason Rohrer

Independent Games Summit

Game Developers Conference 

March 29, 2018
\end{center}


{\Huge



\begin{itemize}

\item Motivation [Slide:  Shia Laboeuf]

\item Just Do It is an instantaneous action, not a process.

\item Difference between momentary motivation and long haul productivity.

\item Part of being independent is that we're trying to escape from this [Slide:  office hell]

\item All those things that go along with ``business'' seem like hell.

\item Rigid schedules, deadlines, a strict separation of work and non-work, etc.

\item We're doing this to ENJOY it, right?

\item We can do it however we want [slide:  My Bed]

\item We can have flexible schedules, keep odd hours, work only when we want to work, interleave work and non-work throughout the day, and so on.

\item This might be enjoyable, for a while.

\item But in the end, shipping a game takes a lot of hard work.

\item And it IS our job, if we want to keep doing it over the long haul.

\item A rhetorical question with no show of hands:

\item How many people in this room, at some point in their careers, failed to ship their game and had to go back to work?

\item There's nothing ``enjoyable'' about that.

\item These business guys aren't idiots [Slide: office hell again]

\item If they could squeeze extra productivity out of their teams by letting them work in bed, they'd do it in a heartbeat.

\item They'd install a bed in every cubicle.

\item This guy is looking at his watch.

\item A classic over-scheduled business thing that we want to avoid, right?

\item Time is money, and all that crap. [Slide: time is money]

\item But even if we don't care about maximizing money per unit of time.

\item Time is LIFE [Slide].

\item You only have enough time in life to ship a limited number of games.

\item Central thesis:  making games is our job, our business.

\item So we should behave like we are working.

\item We should manage ourselves the way we would manage an employee.

\item And like a business, we can start by taking an inventory of past performance.

\item I've done this for myself, and I'm going to walk through it now.

\item I did this for all 19 games, but here are some highlights. [Slide:  career highlights].

\item  Most well known game, most commercially successful game, current game.

\item Before inventory, I knew these figures [Slide: dollars per game]

\item Projects also getting more complicated [Slide: LOC per game]


\item And dev time getting longer [Slide: Weeks per game]

\item But what about dev productivity [Slide: LOC per week]

\item And what about financial productivity [Slide: Dollars per week]

\item Version control and detail [Slide: git log]

\item Extremely detailed view of entire career  [Slide: loc per week career]

\item Binned by month  [Slide: loc per month career]

\item And annotate it  [Slide: loc per month career, colored]

\item Hard to do a detailed productivity analysis of ancient history.

\item Going to focus on just the most recent project [Slide:  OHOL LOC per week]

\item Vacations [Slide: OHOL LOC Overlay]

\item Conferences [Slide: OHOL LOC Overlay]

\item Startup [Slide: OHOL LOC Overlay]

\item Slump [Slide: OHOL LOC Overlay]

\item Tom Joins [Slide: OHOL LOC Overlay]

\item Adjust [Slide: OHOL LOC Overlay]

\item On Fire [Slide: OHOL LOC Overlay]

\item What happened? [Slide: OHOL LOC Overlay]

\item Tom Leaves [Slide: OHOL LOC Overlay]

\item Regroup [Slide: OHOL LOC Overlay]

\item Test launch [Slide: OHOL LOC Overlay]

\item Insane crunch [Slide: OHOL LOC Overlay]

\item Nervous and physical breakdown [Slide: OHOL LOC Overlay]

\item Rebirth [Slide: OHOL LOC Overlay]

\item Rest of talk will focus on last two sections:  Crunch vs productivity rebirth. 

\item Insane crunch looks like the biggest burst of productivity in the whole project.

\item But it literally almost killed me.

\item Let's look at some other stats to figure out what happened.

\item Lifetime hour of day productivity [Slide: hour of day]

\item Crunch is even worse [Slide: crunch hour of day]

\item Lifetime day-of-the-week productivity [Slide: day of week]

\item Crunch is even worse [Slide: crunch day of week]

\item So, how did I fix it?

\item Let's return to basics:  How to ship a game [Slide: how to ship a game]

\item Assuming prerequisites:  [Slide: ship assumptions]

\item Shipping a game is pretty simple:  [Slide: ship method]

\item The hard part is actually doing this, and continuing to do it, over the long haul required to ship a game.

\item Independent video game dev is rather new

\item What long-standing creative practice can we compare it to?

\item Writing [Slide:  Dahl stats]

\item How did he do it? [Slide:  Dahl audio]

\item He did two hours in the morning, and another two hours in the evening, for a grand total of four hours a day.

\item Sounds like he was strict with himself about stopping.

\item So that's tool number 1 [Slide:  Limited work hours]

\item Some details [Slide: Limited hours details]

\item Tip:  You have the rest of the day to do that other stuff.

\item Tip:  Keep a non-work list.

\item Why is quitting the hard part?

\item It's hard to quit because of Flow [Slide: flow]

\item One of the hallmarks of flow is losing track of time [Slide: clock]

\item Flow fallacy [Slide: Flow fallacy]

\item Flow as a crutch to keep going, avoid procrastination.

\item Perhaps not needed if you have other ways to stay on track.

\item You want to consciously chose how you spend your time.

\item You want to pay attention to how much time the current task is consuming.

\item And that's where this comes in [Slide:  timer]

\item Pomodoro Technique summary [Slide: pomodoro cycle]

\item Paying attention to where your time is going.

\item Every 30 minutes, you wake up out of flow.

\item Is what you're working on worth the time you're spending?

\item Even more important [slide:  bad sitting]

\item So that's tool number 2 [Slide:  regular breaks]


\item A few words about stopping.  

\item If you've really been super productive for 5 solid hours, you can and SHOULD stop.

\item Do other things besides work each day.  Have hobbies.  Spend time with family and friends.  Sleep and eat well.  Dream. [Slide:  drone racing]


\item Games are some of the most complicated engineering projects in existence [Slide: camera diagram]

\item How to organize your process [Slide: clipboard]

\item First, why paper?  Why not digital?

\item It's actually WAY FASTER.

\item Also:  Can draw diagrams, cross stuff out, annotate stuff.

\item NOT a notebook.  Loose 3-hole-punched sheets

\item Can flip through pages quickly, reorder pages at will, etc.

\item The power of the stack.

\item Can pull out one page and put it on your desk while you work.

\item Clipboard prevents a mess.

\item Finished notes on bottom, special section.

\item Eventually move finished notes to a folder or 3-ring binder.

\item Tool number 3 (loose sheet paper notes on a clipboard)





\item If you do these three things consistently, you will be able to efficiently ship as many games as you want to ship while maintaining mental and physical health.

\item The problem of starting a method and not sticking with it [Slide: nutrition search]

\item Initially, the new method will create a temporary renaissance.

\item You know you want to, you've decided to, but old habits creep in.

\item Then you feel guilty, and eventually ditch the method.

\item Another trick from a writer, this time from Jerry Seinfeld.

\item Don't break the chain [Slide: my chain calendar]

\item Hang in a visible place.

\item Give yourself a red X for every day you kept the habit.

\item Try to keep the chain going as long as you can.

\item So, that's the method.  Stupid simple.  [Slide: method summary]

\item Finally, how did it work?  Last 4 slides comparing crunch vs. rebirth.



\end{itemize}

}

\end{document}
