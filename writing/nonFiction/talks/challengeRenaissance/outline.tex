\documentclass[12pt]{article}
\usepackage{fullpage,times}

%\pagestyle{empty}


\begin{document}

\begin{center}
The Challenge Renaissance

Jason Rohrer

Montreal International Games Summit

November 2, 2011
\end{center}


\section{The Mainstream Problem}
{\Huge



\begin{itemize}


\item CODmw2 (Nov 2009): 20 million units sold, brought in \$1.2 billion

\item Worldwide, this revenue is competing with the top 3 films of all time. [Slide: boxoffice chart]

\item Super-popular game, huge impact on culture, right?  Everyone knows about?

\item  Ask people on the street.  Neighbor.  Aunt.

\item Tons of money... but no one has heard of it.  Why's that? 

\item 20 million units seems like a lot [Slide:  units graph 1]

\item Avatar: 348 million units [Slide: graph 2]

\item Titanic: 400 million units  [Slide: graph 3]

\item Hard to get exact figure (70 years old) but

\item Gone with the Wind sold something like a billion units [Slide: graph 4] 


\item World population 1939: 2 billion. [Slide: graph 5] 


\item Demographic:  males 15-35
\item CA 2010 census:  15\% of the total population


\item Don't celebrate conquest of mainstream just yet.

\item Big part of mainstream problem involves journalists.


\item Mainstream video game journalism is divided into two camps:

\item Drooling fanboys, conned way into mainstream press, compare Zelda to Citizen Cane.  

\item Pretty much write about nothing but video games. [Slide:  Seth Schiesel quote]  Shi-ZELL (like shi-ZAM + gazelle)

\item Luddite oldsters, dip their toes in water, write humorous accounts of general befuddlement.  

\item Each writes one article about video games, moves on.  [Slide:  Nicholson Baker quote]

\item And then we have Tom Bissell.  [SLIDE: Tom Bissell]

\item Born 1974.  Straddles cultural divide.  

\item Wrote New Yorker article about video games.  Writes whole book about video games.  [Slide:  Extra Lives]  

\item Good book.  Mainstream book.  Book gets good reviews.  

\item NY Times reviewer compares Bissell to Marshall McLuhan of 50s and 60s.

\item And doesn't just write about video games.  Writes about real world too

\item About Soviet Union, about Vietnam [Slide: Father of All Things].

\item We need Tom Bissell  [Slide:  WE NEED TOM BISSELL]

\item We just lost Tom Bissell:  [Slide:  Bissell quote 1] [Slide:  Bissell quote 2]



\item Bissell not an isolated case.  

\item All my intelligent, thoughtful 30-something friends have made same promise to themselves.

\item Done playing video games, after lifetime of playing.  

\item Have wives and kids now.  

\item But new priorities leave room for movies and books.

\item No one that I know plays video games anymore, 

\item Except game designer friends in industry.


\item Not relevant to these people's lives anymore.


\newpage
\item Brings us back to tired, old challenge [Slide:  line in the sand] {\large \begin{itemize}
\item {\it La Lecon} (1951) by Eugene Ionesco
\item {\it Lolita} (1955) by Vladimir Nabokov (nuh-BOE-kof)
\item {\it Taos, NM} (1947) by Henri Cartier-Bresson
\item {\it Blue Velvet} (1986) by David Lynch
\item {\it Guernica} (1937) by Picasso
\item {\it Model of the Monument to the Third International} (1920) by Vladimir Tatlin
\item {\it Sgt. Pepper's Lonely Hearts Club Band} (1967) by The Beatles
\item {\it Shadow of the Colossus} (2005) by Fumito Ueda
\item {\it The Legend of Zelda} (1986) by Shigeru Miyamoto
\item {\it Metal Gear Solid 2} (2001) by Hideo Kojima 
\end{itemize} }

\end{itemize}




\section{In-lined Microtalk}

\begin{itemize}

\item Figure out how to cross line, get there

\item Indie types working really hard on this.

\item Figure out what keeps mainstream away.

\item Shed old, annoying design patterns.

\item Meanwhile, add content sophistication.

\item Artgames, expressive games, whatever you call them.

\item Games like these resulted [MTSlide 1:  boring games]


\item Lots of people liked these

\item Games like these are reason I'm standing up here.

\item BUT I'm not satisfied with result.  

\item Come to detest these games, many ways.

\item Not just me.  Artgame movement is pretty much dead.

\item But WHY?  Spent lot of time thinking about this.

\item Conclusion:  these games are boring.

\item Realization: [Slide:  S+S EP] not enough to just wander around incredibly beautiful environment and do stuff

\item Realization kindof startling

\item Interactivity is not enough!  [Slide:  slogan]



\item Starting thinking about boredom itself.

\item May be hard-wired.

\item Survival mechanism to avoid unproductive situations

\end{itemize}

\noindent [MT slide 2:  shooter]
\begin{itemize}
\item These games aren't filled with tense action.

\item Action isn't the only gameplay not boring.

\item And lots of people say they aren't fun.

\item Forget about fun.  Lots of things aren't fun but still not boring.

\item But it doesn't matter how profound our expression

\item we're losing the audience through walking out or drifting out due to boredom.

\item we need to engage people!  Keep them there.

\end{itemize}

\noindent [MT slide 3:  Bioshock Shrugged]
\begin{itemize}

\item and we don't want gimmicks that fight the expression.

\item not a spoon full of sugar

\item fundamentally delicious medicine.  How to do that?
\end{itemize}

\noindent [MT slide 4:  painting]
\begin{itemize}

\item Got me thinking about other mediums

\item How do they handle engagement?

\item How long can they engage?

\item How long can we stay focused without drifting?

\item Durational tolerance.

\item How long to spend looking at a painting?  30 seconds.

\item Music engages longer.  Song lasts several minutes.  Some even longer
\end{itemize}

\noindent [MT  slide 5:  music mosh]
\begin{itemize}

\item Only engages ears, rest of mind/body free.

\item Driving, cleaning, cooking.

\item I was listening to music as I wrote this talk.

\item Experiment:  listen to music with no parallel activities.

\item Stare at the speakers.

\item Hard to focus for more than few minutes.
\end{itemize}

\noindent [MT slide 6:  Rosemary's Baby]
\begin{itemize}

\item Movies engage eyes, ears, and good portion of brain.

\item Parallel activities limited.  Knitting.

\item Tolerate movies much longer.  Several hours.

\item Richer sensory engagement increases tolerance?  Is that it?
\end{itemize}

\noindent [MT slide 7:  Play Time]
\begin{itemize}

\item Clearly not the case, because some movies are boring.

\item Engaging eyes and ears, by itself, not enough.

\item Content matters!

\item But what content?

\item Answer is disappointing:  magic ingredient is plot.
\end{itemize}

\noindent [MT slide 8:  Plotless vs Plotted]
\begin{itemize}

\item Avant garde used to resist plot.

\item They gave up.  Today's most boundary-pushing art films embrace plot.

\item Maybe experimental plot, but still plot.

\item So... why is plot so engaging?

\item Again, may be hard-wired for survival.
\end{itemize}

\noindent [MT slide 9:  crowd vs. argument]
\begin{itemize}

\item City park bench experiment.  Mind wanders.

\item Couple argument.  Something going down:  whoa!  Can't stop watching.

\item Might learn something.  Future social or survival advantage.

\item If something's going down, you can watch closely for several hours.

\item Something going down = plot!
\end{itemize}

\noindent [MT slide 10:  Alice's Restaurant]
\begin{itemize}

\item Add magic ingredient to music, can pay attention much longer.

\item Arlo Guthrie:  ``Alice's Restaurant'', 18 minutes long.

\item Can listen to whole thing, no drifting.

\item No parallel activity needed.

\item Why?  Song has a plot.
\end{itemize}

\noindent [MT slide 11:  Arcade cabinet]
\begin{itemize}

\item Apply line of thought to games.

\item Game engage eyes, ears like movies.

\item Also engage BODY (interactivity).

\item Require active consumption.

\item Monopolize more of your brain than any passive medium.

\item No parallel activities possible.  Can't even knit!
\end{itemize}

\noindent [MT slide 12:  gamer dad]
\begin{itemize}

\item Multi-channel engagement.

\item Durational tolerance goes from a few hours to a few dozen hours.

\item Single play sessions:  four hours is common.

\item Everyone has blown a whole day playing one game.

\item We lose sleep over games---that engaging.

\item Who would remain engaged by a movie for a dozen hours straight?

\item Plot just isn't powerful enough.
\end{itemize}

\noindent [MT slide 13:  Plot in games]
\begin{itemize}

\item Plot not enough in games, either.

\item So what ingredient leads to massive durational tolerance?

\item My only answer:  Challenge.

\item Again, may be hard wired engagement response.  Survival value.
\end{itemize}

\noindent [MT slide 14:  Challenge in games]
\begin{itemize}

\item Passively watching something going down.

\item Might learn something.  Pay attention for a while.

\item But if something is going down that involves you?

\item Something you've gotta overcome?

\item Survival value is obvious!
\end{itemize}

\noindent [MT slide 15:  two arguments]
\begin{itemize}

\item Can watch someone else's argument passively for a few hours.

\item Personal argument?  All day!  Days!  Months!

\item As long as it lasts.

\item That's what challenge does.  Wow.

\item Not advocating for long games.

\item But it is interesting that they CAN engage so long.

\item Monopolize all channels, brain DEMANDS challenge.
\end{itemize}

\noindent [MT slide 16:  Boring passage]
\begin{itemize}

\item Even for short sessions.

\item Even for a five minute game.
\end{itemize}


\noindent [MT slide 17:  artgames vs. challenge]
\begin{itemize}

\item Work monopolizes eyes and ears (watching), brain yearns for plot.

\item Work demands that you actively DO something, brain years for activity to be challenging.

\item We've been fighting against necessity of challenge in games.

\item Hopeless:  like fighting against plot in movies.

\item Disappointing!  Challenge is old-fashioned and limiting.
\end{itemize}

\noindent [MT slide 18:  Synecdoche, NY]
\begin{itemize}

\item Filmmakers have embraced plot, not too limiting for them.

\item Huge variety of plots.

\item Expression woven through these plots.
\end{itemize}

\noindent [MT slide 19:  Mario]
\begin{itemize}

\item Our problem:  limited spectrum of game challenges invented so far

\item Jumping, shooting, racing, puzzles.

\item Weave expression through those?  Not cohesive.

\item Challenge feels like a spoon full of sugar.
\end{itemize}

\noindent [MT slide 20:  Doom vs. Bonnie]
\begin{itemize}

\item Imagine if we just knew one movie plot:  Bank Heist

\item Imagine filmmakers trying to weave expression through Heist movies.

\item Would feel limiting!

\item Would want to abandon plot to circumvent limit. 

\item FPS and TPS is like a movie where main character must carry a gun the whole time.

\item But the solution for us:  not throw out challenge.

\item We need to INVENT new challenges that complement what we want to express.

\item Essentially means inventing new genres.

\item Challenge will stop feeling like engaging gimmick

\item Will be a potent expressive tool in own right.
\end{itemize}



\begin{itemize}

\item Several designers and critics have complained about how we cling to ``games'', like this is historical baggage to be dumped.  [Slide:  TOT Notgames]

\item Holding this interactive medium back.

\item Not reaching full potential.

\item Their sentiment is dead wrong.

\item Games dominate ``interactive art'' for a reason.

\item We subconsciously gravitate toward making games because it's only way to make non-boring interactive art.




\item Minecraft often described as sandbox game [Slide:  Minecraft / Lego]

\item Many say Minecraft is like digital box of legos.



\item Build whatever you can imagine [Slide:  Cathedrals]


\item But that can't explain humongous appeal.

\item Tons of digital tools/toys for building whatever you can imagine.

\item No, minecraft is not like digital box of legos.




\item It's like box of legos that ships with live rat inside [Slide:  Mincraft rat]


\item Rat comes out at night and tries to bite your fingers as you work.

\item It's this aspect of Minecraft---incredible challenge you face when trying to build what you imagine---that gives Minecraft appeal.

\item Brings us back to boring, challenge-free artgames [MT slide 1: boring art games]

\item We rejected challenge.

\item Not just because it is old fashioned, though.

\item Because we felt that challenge was driving away diverse audience.

\item My wife, my uncle, Roger Ebert, etc.

\item Making sophisticated games for non-gamers.

\item Non-gamers can't tolerate challenge, right?

\item But perhaps challenge is scapegoat here.



\item My first experience with Kane and Lynch 2 [Slide:  K+L2 title]

\item Inspired by Tom Bissell, started on hardest difficulty.

\end{itemize}

\noindent Run through 3 times:
\begin{itemize}
\item Run through the beaded curtain (Lynch: ``I want him alive'')

\item Out onto the balcony (Lynch: ``I just wanna talk'')

\item Through someone's apartment (Kane: ``I see that you're still insane, Lynch'')

\item Blocked by an iron gate (Lynch: ``Brady goddamit'' );

\item Through someone else's apartment (Lynch: ``Is there a back way in here'')

\item Out the back door and get shot dead (Lynch:  ``Shit'').
\end{itemize} 

\noindent After about six more repetitions, I quit the game and restarted on medium difficulty.


\begin{itemize}


\item So not challenge driving away normal people.

\item It's broken, repetitive way we handle failure, mainstream game design.

\item No reasonable person wants to repeat the same damn thing over and over!

\item But in linear game, what else can you do?

\item Mainstream design has noticed this.

\item Checkpoints have gotten closer and closer together.

\item Repetition reduced.




\item Limit of this trend is BioShock [Slide: BS vita chamber]

\item Eliminated failure repetition from game completely.

\item Die, go back to last vita chamber SPATIALLY only.

\item VitaChamber not checkpoint---just physical jump-back point.

\item World state does not rewind when this happens.

\item Dead enemies still dead.

\item Injured enemies still injured, including bosses.

\item Sweet!  never repeat anything!

\item Weird side-effect:  challenge eliminated too!


[Slide:  bioshock wrench]

\item Example:  beat the first boss with pipe wrench.

\item I did this, no strategy, walk forward, mash trigger, respawn.

\item Took 20 respawns, but first boss fell.

\item A somewhat engaging game becomes BORING when you figure this out.

\item Game is FALSELY engaging.




\end{itemize}


\section{The Challenge Renaissance}
\begin{itemize}
\item There have been other approaches to dealing with this problem.

\item They embrace challenge fully.

\item So fresh and invigorating.

\item Most importantly, NOT BORING.

\item Hard to not see them as the future of games.

\end{itemize}



\noindent Aproaches:

\begin{enumerate}




\item Freedom of Approach  [Slide: FC2, Demon's Souls, Minecraft]

Hallmarks
\begin{enumerate}
  \item Extremely challenging
  
  \item Systems heavy

  \item You pick what you're trying to do.
   
  \item Die in the middle of what you're doing.

  \item Respawn WAY BACK (effectively no checkpoints)

  \item So far back, you might as well pick something else to do next

  \item Or approach what you were trying to do in a totally different way.

  \item (Whereas linear games make you try approaching by river with a shotgun over and over from the last checkpoint until you succeed, FC2 lets you try the back entrance on foot with a sniper rifle next time---each approach to the same challenge feels fresh)
\end{enumerate}




\item Roguelike [Slide: Spelunky, Shootfirst]

Hallmarks
\begin{enumerate}
  \item Extremely challenging.

  \item Permadeath, including entire game world and all it's unexplored possibilities.

  \item Each new attempt is a brand new game world with new challenges waiting.
\end{enumerate}





\item Microchallenge [Slide: Flywrench, VVVVVV, Super Meat Boy]

Hallmarks
\begin{enumerate}
  \item Extremely hard, small chunks of challenge.

  \item Chunks big enough that they require mastery to pass.

  \item Explicit checkpoints or boundaries between challenges.

  \item Instantly respawn to try again upon failure.

  \item Practice your current failure spot dozens of times in one minute.
\end{enumerate}

\end{enumerate}


\begin{itemize}


\item All of these generally share the following important property (which is NOT true of my K+L2 example):  no repetition of linear content!  Seeing the same enemy run to the same spot again or hearing the same quip from your partner.  Repeating that stuff is grating as hell.


\item These approaches are not at all compatible with linear, pre-authored stories.

\item These games feel very pure as games.

\item They also feel somehow, strangely, expressive.

\item Like a new, alien form of expression.  An expression born out of challenge itself.

\item The Microchallenge games, in particular, seem to be expressions about the nature of challenge itself.


\item And that brings us back to Tom Bissell [Slide:  Bissell photo]


\item Not leaving games because too challenging, or because sick of pointless challenge.

\item On contrary, always plays hardest difficulty setting.

\item What he's sick of is bullshit content [Slide:  Bissell Quote]


\item So keeping Tom Bissell?

\item Might involve preserving challenge and dispensing with rest.

\item And exploring challenge, deeply, is what I believe need to do with games artistically.  

\item Boundary we need to push as artists.  

\item That's how we get to where we want to go.  

\item Challenging games = culturally sophisticated games.  

\item Can't push artistic boundaries by pandering to game-illiterate mainstream.  


\item Grandma may not be doing Far Cry 2 permadeath runs [Slide:  grandma playing Wii]

\item She's not watching Lars von Trier films either

\item But where does that leave us?  

\item Trapped in our niche without the mainstream.

\item But its only way we're ever going to get across that boundary [SLIDE: line in the sand]





\item And this seems like paradox.  

\item To transcend into artistic achievement:  must forget about appealing to EVEN the INTELLECTUAL mainstream.  

\item Even the New Yorker readers and von Trier watchers.

\item Doesn't seem like films had to do that, why should we?  

\item Why can't we have both?

\item Chris Hecker has pointed out:  comics screwed up, 

\item Even the artisically relevant graphic novel can't save comics.  

\item Too late for comics to shake off throwaway-culture image.

\item But I think he's WRONG about that.


\item Because reading graphic novel, especially artistically ambitious one, like something from Chris Ware [Slide: Chris Ware page]

\item A lot of work.  

\item Not a passive experience.  

\item Must try, unpack it, figure it out.  

\item Eyes don't always know where to look next.

\item Not relaxing.  Not something to do after hard day at office.  

\item And just try reading one to your kid!  Trying to read graphic novel out loud is painful.

\item This is not true for books, movies, or music.

\item Essentially, no way to casually consume even a partially-active medium like graphic novels.

\item And here we are making works in the most active medium ever invented.









\end{itemize}
}

\end{document}
% LocalWords:  Minecraft
