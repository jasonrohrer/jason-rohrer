\documentclass[12pt]{article}
\usepackage{fullpage,times}

%\pagestyle{empty}

\begin{document}

{\Huge

\begin{itemize}

\item Games have a major legitimacy problem.  Cultural ghetto.  [Slide:  line in the sand] {\large \begin{itemize}
\item {\it La Lecon} (1951) by Eugene Ionesco
\item {\it Lolita} (1955) by Vladimir Nabokov
\item {\it Taos, NM} (1947) by Henri Cartier-Bresson
\item {\it Blue Velvet} (1986) by David Lynch
\item {\it Guernica} (1937) by Picasso
\item {\it Model of the Monument to the Third International} (1920) by Vladimir Tatlin
\item {\it Sgt. Pepper's Lonely Hearts Club Band} (1967) by The Beatles
\item {\it Shadow of the Colossus} (2005) by Fumito Ueda
\item {\it The Legend of Zelda} (1986) by Shigeru Miyamoto
\item {\it Metal Gear Solid 2} (2001) by Hideo Kojima 
\end{itemize} }



\item something about VIDEO games makes us think about them differently than we think about board games.... the whole premise of this conference.  [Slide: VIDEOgames]

\item Maybe because they're on screens? [Slide: movie hearts]


\item We readily admit---if not demand---that games are different from movies, somehow.  But how?  

\item Board games and film coexisted for about 90 years without anyone ever pointing them out as being similar.  Even today, we don't make the boardgame/movie connection.  

\item But we think about video games differently, as I said before.  We don't think about them in terms of rules and systems.  

\item We talk that way as designers... but in big picture terms, as far as the medium is concerned, it's all about something else. ``Interactivity.''  [Slide: ``Interactivity'']

\item So then we stumble into a kind of default mode of thinking about video games.  They're like movies, but interactive.  [Slide: interactive movie]

\item And how do movies explore the human condition and do other art-type stuff?  By telling stories.  Thus, we think about video games becoming art-like, we think of them as a storytelling medium.

\item As designers, we generally *hate* that games have moved in this direction.  We hate cut scenes and other movie-copycat techniques.

\item But we're not really complaining about the goal.  More like we're thinking, ``Wait, you're doing it wrong.''

\item Because, you see, many of us secretly have dreams about cracking the real problem---how to make interactive movies without borrowing any of the non-interactive baggage from movies.  

\item How to make ``totally interactive movies''. [Slide: totally interactive movie]


\item These ideas usually involve a ton of hand-waving.  

\item Setup systems of emotion and relationship mechanics and let them churn against each other and somehow, through a really intricate mechanical webwork, some kind of interesting, interactive story will emerge.  [Slide:  gears]

\item Call this the {\it clockwork} approach.

\item The other approach involves a similar amount of hand-waving.  Set up an AI for each character, and another AI to ``direct'' the whole thing and maintain the story arc, and let them all churn against each other, and somehow, some kind of interesting story will emerge.  [Slide:  HAL] 

\item Call this the {\it Hal} approach.

\item In both cases, we're banking on emergence.  We love emergence---it's like a game designer religion.

\item But when the rubber hits the road, it's really hard to envision how either approach would end up producing interesting stories.

\item For whatever reason, this has been mostly vacuous hand-waving, and no one has spent too much time on it.  

\item Except Chris Crawford, who sunk 17 years into the clockwork approach with negligible success. [Slide: dragon speech charge]

\item And while he was doing that, sure, a number of us thought he was a little nuts or on a road to nowhere.  But far more of us saw a glimmer of hope there... that he would actually be able to do it. Thank goodness *someone* is working on this important game design problem. And wouldn't it be awesome if he did it?  I mean, who wouldn't want to delve into one of these clockwork storyworlds?

\item Let's not forget about the Facade guys, who spent 5 years hammering away at the Hal approach. [Slide: Facade].


\item At least they released something, and it was loads of hilarious fun to play, but it failed at it's primary goal of making characters that respond in believable ways, even if you play totally straight and go along with the story.


\item But Facade (july 2005) was enough to pack more opium into our dream pipes.  Shine light on the incredible mountain before us.  Show us a faint path through the foothills.


\item And this is where we start talking about our reasons for failure, so far.

\item David Jaffe's failed Heartland project.

\item A quote about Heartland [Slide: Jaffe quote 1]

\item More quote [Slide: Jaffe quote 2]

\item Chris Hecker has described what he calls ``the grain of a medium'' [Slide: Hecker quote 1]

\item Easiest game to make vs. the hardest game to make.

\item So what are we facing, according to Hecker?  [Slide:  Hecker quote 2]


\item Essentially, we don't know how to do this yet.  We're kinda stuck for the time being.  Chris Crawford has been working on this for 17 years, and we still have 20 more years to go.

\item 20 more years before we can start making emotionally compelling art.  Yikes.

\item Okay, but baby steps.  We're not going to take a nap until Ray Kurzweil's AI singularity, are we?  

\item First, let's drop dialog, because for the time being, making dialog work well interactively is impossible.  

\item Let's design simple systems and tell simple interactive stories with them.  

\item Why am I here?  Because I made a dialog-free, totally-interactive game that many people found to be emotionally compelling [Slide: Passage].  

\item Wow, people said, I love the story that was told *through* the game mechanics in Passage.

\item Another example of this can be found in Far Cry 2 [Slide: FC2]

\item FC2 contains emotionally-evocative situations, similar to what you might see in HL2, but without any of the on-rails, scripted stuff.  

\item You *can* make meaningful choices within the system of FC2.

\item And the game of the year, 2009?  Derek Yu's Spelunky [Slide: Spelunky].  

\item As you play, your choices, meshed with emergent properties of the mechanics, lead to a gripping, interesting, always different story about a guy exploring a dangerous cave.  

\item Sometimes, the situations you encounter and the follies that lead to your death speak directly to the human condition.  

\item Spelunky is funny, but funny because it's true.

\item Sure, there aren't really interactive *characters* in these games.  

\item They're mostly about physical objects, and navigating spaces, and perhaps a mechanical metaphor or two.  

\item But hey, at least games like these are Totally interactive and consistent, and at least they touch on the human condition.  I guess they're the best we've got.  Baby steps on the stairway to Chris Hecker's pantheon.



\item I neglected to mention one more Interactive Storytelling game.  Masq [Slide: Masq] (2002, major coverage in 2007), which also consumed 5 years of development time and slowly rose to popular attention 5 years after its release.  

\item Most of us played it after Facade.  

\item Totally different approach.  Not clockwork or Hal---just a simple, branching story web, like a very fine-grained choose-your-own adventure book

\item but it actually worked, which was really weird.  

\item And in the process of actually working, it pissed a lot of people off, because it worked without making any progress on the technical holy grail problems that we're all dreaming about.

\item Yes, the Masq characters responded in realistic ways to your every action.

\item Yes, you could pretty much do whatever you wanted most of the time (5 disparate choices per mini-node is a heck of a lot).  

\item But there was no illusion there.  These were just pre-authored characters baked into a story web.  

\item They weren't artificial people, clockwork people.

\item And that's what we want, when it boils down to it.  THAT is what we're dreaming about.  Clockwork people. [Slide: humanoid].  

\item We see that the AI world has failed at its 50-year windmill quest.  

\item We see that the robotics world has failed at its own centuries-old quest. [Slide: turk]  

\item We see a glimmer of hope in our medium to help make clockwork people.


\item But why?  Why would anyone want a mechanical turk?  

\item Note that throughout this entire talk, through all the games discussed, and all the quotes, we have never once mentioned something, and it's a strange omission, given that we're talking about games.  

\item We've never once talked about the *OTHER PLAYER*.  [Slide: chess playing men]

\item What *IS* the grain of our medium?

\item Sit down with some friends around your kitchen table.

\item What's the easiest game to make?  One where you talk to each other a lot, negotiate, promise, befriend, betray, and so on. [Slide: werewolf]

\item What's the hardest game to make?  One where spaceships fly around and shoot things. [Slide: starwars board game]

\item Part of our problem is again connected to how we think about movies.  

\item A movie is a little self-contained box that can operate in isolation.  

\item Bring it home, open it, pop it in the player, and watch.

\item It's all in there.  The characters, the dialog, everything.  Pre-baked.

\item We neglect talking about OTHER PLAYERS when we talk about the artistic future of games because we want our artistic games to be little, self-contained boxes, too.  

\item When we talk about characters in games, we want a little box full of interactive characters.


\item So I think these hearts are flowing toward the wrong target. [Slide:  hearts toward movies].  

\item If we really want to make games that deal with the human condition.  

\item If we want to make games that have meaningful, interactive characters.  

\item And if we want to do it NOW, without waiting for some hypothetical strong AI singularity in the distant future. 

\item We should be looking in a different direction entirely.  [Slide, hearts toward theater].



\end{itemize}
}

\end{document}