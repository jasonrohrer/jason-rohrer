\documentclass[12pt]{article}
\usepackage{fullpage,times,verbatim}

%\pagestyle{empty}


\begin{document}

\begin{center}
Frog Skin

Game Design Challenge

Jason Rohrer

Game Developers Conference 

March 7, 2012
\end{center}


{\Huge



\begin{itemize}



\item Back in 1928, this was MONEY [hold up 1928 bill]

\item 1928 \$20 Federal Reserve Note

\item But THIS was also money back then [1900 gold double eagle]

\item \$20 US coin, contains about 1 oz gold

\item (Read Note inscription).

\item You could walk into the treasury, give them one of these (bill), and get one of these (coin).

\item On April 5, 1933 that changed forever [slide:  Executive order 6102]

\item They'd still be happy to give you one of these (bill) for one of these (coin), but not the other way around.







\item Strangely, US STILL minting 1-oz gold coins, like this one [Show modern 1990 Eagle]

\item But you can't walk in with one of these \$20 bills and buy one [Show 1928 bill]

\item These days, you need a huge stack.  90, in fact [Fan stack of 90 moder 20s]

\item We've all seen examples like this [Slide:  bike for \$9].


\item Inflation is so old and constant now

\item Just an accepted fact of life.

\item Part of the reason we accept it is that we've never had this [Slide:  Sweeping up the Hungarian Pengo]

\item Hyperinflation.

\item Money becomes worthless so quickly

\item People throw it away

\item 28 separate occurrences past 100 years around world



\item US never experienced this

%\item So we're not too worried about it

\item Haunted by a different specter, Deflation [Slide:  great depression]

%\item That lesson was strong, still talk about it today.

\item So scary that we've allowed no major deflation since then [Slide:  US Inflation].



\item Over the past 20 years, inflation never above 5\%

\item This is what most economists would call ``reasonable,'' moderate.

\item Result on prices still dramatic [Slide:  US CPI]

\item Because of compounding, any positive rate gives exponential growth.

\item Prices today are double what they were in 1987.

\item And 22x higher than they were 100 years ago.

\item Again, so used to living with it, don't worry about fixing it.

\item Inflation/deflation = money malfunctions.

\item We have slight malfunction.  Isn't that okay?

\item I say:  NO, it's NOT okay!

\item To understand why, need to think about what money REALLY is.



\item What is this?  [Hold up \$20 bill]

\item This piece of paper?  This particular one?

\item DELAYED CONSUMPTION

\item Me holding this piece of paper says:

\item I produced \$20 worth of something, someone else consumed that something, and I haven't consumed \$20 worth of something else in compensation.  Not yet.

\item I gave into the system without taking out of it.  Not yet.

\item And the ``Not yet'' is the important part. [Slide:  Not yet]

\item Possible to give and take immediately---that's barter.

\item Also possible to give into system and expect nothing in return.

\item Put a basket of surplus tomatoes out at the curb with a free sign.

\item But when you accept money for what you give, you retain the right to take something from the system later, in exchange for what you gave.


\item When I finally spend this \$20, I release my pent-up ``Not yet'' by consuming something that someone else produces

\item In turn, pass on the ``Not yet'' to THAT producer, who has now produced but not yet consumed in exchange.

\item That person holds this \$20 now as proof.

\item A never-ending chain of delayed consumption.

\item Elegant.  Beautiful.



\item Perfectly functioning money allows you to delay consumption as long as you want.

\item Inflation turns this lovely ``Not Yet'' into a hot potato

\item Slowly losing value

\item Pass it on before it's too late.

\item Inflation punishes those who delay consumption.

\item Punishes those who produce more than they consume.

\item Punishes the givers.  

\item What could be worse?

\item Of course, we work around inflation in a variety of ways.

\item But this wastes resources.

\item Makes money less efficient.

\item Governments not motivated to eliminate inflation.

\item For a variety of reasons.

\item So, let's take matters into our own hands.

\item Frog Skin is a two-player collectible card game [Slide:  game title] 

\item Has a measurable positive impact on the problem of inflation.

\item Played with ``cards'' that we've all got already...

\item ...In our wallets.  Played with currency itself.


% Start of basic game rules

\item Each player has secret deck of 5 US bills [Slide:  game start]

\item Flip coin to pick who leads first

\item Walk through game to show rules [Series of slides]

\item Winner keeps all bills [Slide]

\item Game could have played out differently.

\item Could have ended in a tie, then what? [Slide: tie game]

\item Players get their own bills back.

\item Looking at this so far, a number of you might notice something:

\item This game is TOTALLY BROKEN.

\item P2 motivated to switch strategies [Slide:  P2 all 100s]

\item P1 motivated to switch too [Slide:  P1 all 100s]

\item Always tie [Slide:  always tie]

\item Either player switches, that player will lose.

\item Nash equilibrium

\item Not only broken:

\item Does NOTHING to reduce inflation

\item Here's one tiny change, fixes both problems.

\item At any point in game, instead of a player playing a bill [hold up \$20]

\item Allow player to do this:  [Tear \$20 in half]

\item Continue playing with halves like they were two full bills

\item Allow same bill to be played in two different turns.

\item Note that this might extend the game beyond 5 turns

\item If player runs out of bills at end, loses remaining turns.

\item Going back to original setup [Slide:  poor vs. rich]

\item Watch how player 2 can win by tearing money.

\item Suppose they play a few turns, just like before

\item Get to this point [Slide: p1's second win]

\item ... [Slides:  walk through rest of game]

\item Winner keeps all, discards torn bills [Slide:  final state]

\item P2 destroyed \$5, but got net payoff of \$211

\item Money supply reduced by \$5

\item That's \$5 out of \$2.7 trillon [Slide:  reduction]

\item 1.8 ten-billionths of a percent deflation [Slide:  deflation]

\item Makes all other money a little bit more valuable.

\item Measurable!  Every little bit helps.

\item Time for me to put my money where my mouth is... Frank!






\end{itemize}

}

\end{document}