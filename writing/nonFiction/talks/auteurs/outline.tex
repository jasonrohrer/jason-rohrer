\documentclass[12pt]{article}
\usepackage{fullpage,times}

%\pagestyle{empty}

\begin{document}

{\Huge

\begin{itemize}


\item Video games have the potential to become the dominant art form of the 21st Century.

\item Good number of people in this room were {\it born} after the seminal game {\it Doom}.  Never known a world without 3D video games, let alone a world without video games.

\item To people like you, maybe my optimism about the future of the medium is obvious... or maybe it's laughable.

\item Regardless, for you people, video games are like television---cultural oxygen.

\item The question, then, is can we save them from the fate that befell television?  Steer them away from their current destiny of mindless entertainment?

\item Promise, at this point, more than delivery.

\item This is a talk is about one small facet of this problem, and about how we're counting on people from your generation to help us solve it.


\item Games have a major legitimacy problem.  Cultural ghetto.  [Slide:  line in the sand] {\large \begin{itemize}
\item {\it La Lecon} (1951) by Eugene Ionesco
\item {\it Lolita} (1955) by Vladimir Nabokov
\item {\it Taos, NM} (1947) by Henri Cartier-Bresson
\item {\it Blue Velvet} (1986) by David Lynch
\item {\it Guernica} (1937) by Picasso
\item {\it Model of the Monument to the Third International} (1920) by Vladimir Tatlin
\item {\it Sgt. Pepper's Lonely Hearts Club Band} (1967) by The Beatles
\item {\it Shadow of the Colossus} (2005) by Fumito Ueda
\item {\it The Legend of Zelda} (1986) by Shigeru Miyamoto
\item {\it Metal Gear Solid 2} (2001) by Hideo Kojima 
\end{itemize} }

 
\item We were well on our way long ago.  [Slide:  EA ad from 1983]

\item ``We're providing a special environment for talented, independent software artists. It's a supportive environment, in which big ideas are given room to grow.''

\item What has changed?

\item Games have become a large-team medium [Slide: another line in the sand]

\item So, the days of the lone ``software artist'' are gone.

\item Not that large teams is the only direction for modern games to go (Braid was made by a team of 2)

\item But assume, for discussion, since we're talking about mainstream games, that large teams are a given.

\item But large teams aren't necessarily the heart of the problem.

\item Flat, non-hierarchical collaboration.

\item Even small, fully-collaborative team art is hard... look at what happened to Beatles.

\item So how does ``art'' persevere in an industrialized, large-team medium?

\item One answer, from 1950s and 60s cinema:  Auteur Theory.

\item Director as artist, rest of team follows director's lead.

\item Huge impact on film industry, even on today's mainstream Hollywood blockbusters. [Slide:  The Dark Knight]

\item Name above the title.  

\item Recognizable MODERN director names... loads of these, even in main-stream cinema.  [Slide:  list of film auteurs]  22 of them 

\item I know all these folks and can easily name a film or two from each, but I've never studied film or worked in that industry.

\item When we talk about films, we very often talk about the person behind them.
\item Most of these guys have had their names on the poster.

\item Recognizable MODERN game designer names...  okay... so I'm pretty familiar with the industry. [Slide:  insider's list]

\item But who might be known to a game-playing outsider?  [Slide: gamer's list]

\item And who's had their name on the front of the box? [Slide: name on box list]

\item Those ``Name on box'' guys are a bit mysterious---how did they leverage the royal treatment for themselves?

\item Industry resistance to recognizing creators?

\item Check out the Spore website and try to find Will Wright's name there.

\item Check out this box [Slide:  close up of spore box]

\item The guy has been on the cover of Wired, given a TED talk, been profiled by the NYer... but he's not noteworthy enough to name on the box?  

\item And there's no doubt that Spore was his brainchild.  [Slide:  Quote from EA CEO]

\item But they don't hesitate to put a FILM MAKER's name on a game box [Slide: Boom Blox]

\item As game designers, we should find this to be insulting.

\item Justification [Slide:  Second quote from EA CEO]

\item He's implying that it's strictly a business decision.  What about the issue of creative control?  About one person's vision not becoming diluted by team and board-room dynamics?  About giving credit to the author?

\item The name-on-the-box problem is just a symptom of a much bigger problem.

\item Even if we were putting all the noteworthy names on the fronts of boxes, we have a much smaller pool of noteworthy names. [Slide:  back to list of designers]

\item Maybe it's a sheer quantity issue... the youth of our medium?  Fewer games being made than films, so fewer noteworthy people making them?


\item Regarding full creative control, who WOULD you hand the reigns to?

\item Who is our Oliver Stone or our David Lynch?  Or our Nabokov?  The super-creative wells of masterful vision?

\item Why are those types still being drawn to writing books or making movies?

\item Why did Christopher Nolan make Memento as a film instead of as a game?  Why wasn't he drawn to our medium?

\item Bootstrapping problem:  Christopher Nolan grew up watching films that deeply inspired him, but the games he played must not have inspired him in the same way.  

\item We need to ATTRACT auteur-types to our medium.  We need to find and cultivate new ones.  When we find good ones, we need to give them complete creative control.

\item Film industry is doing this all the time---cultivating new visionaries that are discovered in the indie festival circuit.

\item We have an indie festival circuit now too, but the industry ignores it.

\item Indie film case Study:  Jared Hess [Slide:  Napoleon Dynamite]

\item Interviewed him back in July 2008.

\item Started with a short (Peluca) at Slamdance 2003.

\item Got \$200K financing from friends to turn it into a feature.

\item Feature accepted to Sundance 2004.

\item Sold after second Sundance screening to Fox Searchlight, the first studio that they met with

\item Other studios were interested, but Fox grabbed it first.

\item Signed a 2-picture contract with Fox.  Napoleon Dynamite, and one of his next two movies released by Fox.

\item Jared's original Sundance cut was used with no re-edits. (He had final cut rights)

\item Shown on 800 screens nationwide, made \$45 Million, cost \$400k to make. 

\item Directed Nacho Libre with Paramount after Napoleon Dynamite.  

\item \$30 million budget, major star involved

\item he didn't have final cut rights, but felt that the studio preserved his vision anyway.

\item Third film had to be made with Fox Searchlight (according to contract).

\item They green-lighted his script for Gentlemen Broncos (Coincidentally, will be released on Friday [October 30, 2009])

\item Budget around \$10 million

\item He has final cut rights.

\item On the set, he has complete creative control.  

\item Every member of the team answers to him and strives to fulfill his vision.

\item Question:  What if the DP is suggesting a shot angle that you don't like?  Can you override him?

\item Answer:  Absolutely.  Everything is my call.

\item So this is the Hollywood picture:  some unknown kid from Utah makes one quirky hit film and is handed \$10 to make whatever he wants.  He has complete creative control of the end product.


\item Flip-side:  Indie Game case study:  Jonathan Blow

\item Braid WON the Design Innovation award at the IGF in 2006.

\item No studios came up to him and expressed interest.

\item A few years later, as Braid neared completion, a guy from MS called and suggested that Jon submit it through the XBLA review process.

\item Braid accepted with a pretty good deal.

\item Jon called Sony also about a PSN release, but not interested.

\item The people MS who originally pursued Braid eventually left.

\item The new people just saw Braid as another product, no different from Frogger.

\item Note that Braid is widely regarded as one of the greatest artistic achievements in our medium.

\item Whether you believe the critics or not, it is still ``award winning'' where Napoleon Dynamite was not.

\item Stark contrast here:

\item Napoleon Dynamite subjected to a feeding frenzy.  Braid virtually ignored.

\item Napoleon Dynamite given a wide national release, just like a ``real'' film.  Braid shuffled off onto XBLA with a bunch of casual games and retro remakes.

\item So that is how the film industry treats a semi-noteworthy work.

\item This is how we treat one of our most noteworthy works.

\item Jonathan Blow may well be the best example that we have of an ``artistic master'' working in the medium of video games.

\item Certainly, we need to treat guys like him and the games that they make better.

\item But people like him are few and far between.

\item There are a lot of gaps to be filled.  A lot of low-hanging fruit.

\item Take virtually any subject of artistic interest---there hasn't been a game made yet about that subject.

\item Even something simple, obvious, and universal, like ``Coming of Age.''  There has NEVER been a game about coming of age.

\item That's where you folks come in.

\item Somewhere in this audience is a young mind with enough creativity to light the world on fire.

\item That person could jump into the crowded pool of another medium, like film, and try to swim.  Maybe be a success, but most likely not.

\newpage 

\item Or that person could, quite literally, become the Orson Welles of games.  Or the Alfred Hitchcock of games.  Or the Woody Allen of games.  Or the Oliver Stone, the Christopher Nolan.  

\item Or the ANYBODY of games, because we don't have an ANYBODY in games yet.

\item There will be names that will go down in history as the people who shaped this new medium, but we don't know those names yet.

\item You could be one of those names.

\item And we need you! We really do.

\item Slide: Please?

  




\end{itemize}
}

\end{document}