\documentclass[12pt]{article}
\usepackage{fullpage,times}

%\pagestyle{empty}

\begin{document}

{\Huge

\begin{itemize}



\item Games have a major legitimacy problem.  Cultural ghetto.  [Slide:  line in the sand] {\large \begin{itemize}
\item {\it La Lecon} (1951) by Eugene Ionesco
\item {\it Lolita} (1955) by Vladimir Nabokov
\item {\it Taos, NM} (1947) by Henri Cartier-Bresson
\item {\it Blue Velvet} (1986) by David Lynch
\item {\it Guernica} (1937) by Picasso
\item {\it Model of the Monument to the Third International} (1920) by Vladimir Tatlin
\item {\it Sgt. Pepper's Lonely Hearts Club Band} (1967) by The Beatles
\item {\it Shadow of the Colossus} (2005) by Fumito Ueda
\item {\it The Legend of Zelda} (1986) by Shigeru Miyamoto
\item {\it Metal Gear Solid 2} (2001) by Hideo Kojima 
\end{itemize} }

\item Partly, this is an external image problem [Slide:  Ebert].  But part of it is a real, internal problem.

\item I've given talks in the past about how we might design games to specifically address this problem.  Advance the medium, make games more culturally relevant, etc.

\item This talk examines the problem itself in more detail:  why are games down there in the first place?

\item Thoughts are structured into two completely ridiculous analogies.

\item Ridiculous 

\item In fact, here's what my friend Frank said about one of my analogies [Slide:  Frank quote]

\item But zooming in, we can see that I was wearing earplugs [Slide:  earplugs]
 
\item Jumping right into the analogies...  [Slide:  analogy 1]

% clever idea... never use the words DRUG or PORN in the talk
% never give away the answer 


\item Observation 1:  value of game often measured in terms of number of hours of gameplay [Slide:  IGN review, Fable II]

\item Short games are worth less to us.  

\item Katamari Damacy debuted in the US for \$20

\item People complained that Braid, a 5-hour game, was overpriced at \$15

\item Despite the fact that it stands as one of the most artistically important video game works of the past decade.

\item What do we want from games?

\item Time killers.  Escape.

\item The more hours of my miserable life a game helps me kill away, the more it is worth to me.

\item 30 hours... is that a good thing?

\item Is the 10-hour alternative in Fable a negative, or a positive?

\item MAN... 10 hours is LONGGGGG.

\item Imagine a 10-hour movie!  A 10-hour record album!  A 10-hour dance performance!

\item In what other medium is value so tied to total duration?

\item Most mediums value conciseness instead.  

\item A 3-hour movie is pushing it, and the director is heckled for being self-indulgent.

\item Observation 2: value of game is often measured in terms of how addicting it is [Slide:  top 10 addicting (mytopcomputergames.com)]

\item Can it hook you?  Can it keep you coming back for more?

\item Flip-side of time killing, because addicting games kill more time.

\item Observation 3: parents worry about their kids playing games [Slide: Immersion Project]

\item Some of this has to do with the addicting and time-killing properties.

\item Is my kid wasting his life?

\item Anecdote about my cousin and RuneScape

\item But there's also a general sense that games are just bad for kids.

\item Consider how state and federal governments keep flirting with restrictions on sale, etc.

\item But once we get out from under our parents' roof, watch out!

\item Observation 4: we all played a lot of video games in college [Slide: dorm players]

\item perhaps more than we should have

\item personal anecdote about late-night Quake mara\-thons

\item personal anecdote about roommate (valedictorian) getting kicked out

\item Note that my dorm-mates and I played Quake fully clothed

% clincher
\item Observation 5:  an overdose can kill you !!!


\item Well, they can't get much worse than that, can they?  Moving on...  [Slide:  analogy 2]


\item Observation:  There are places, mostly hold-overs from previous decades, where you can walk in and play a video game for 25 cents [Slide:  Fun Terminal in SF vs. 25 cent video arcade]

\item normal people don't go into these places.



\item Observation:  These days, there are special stores where people go to buy video games [Slide: game store vs. porn store]

\item normal people don't go into these stores







\item Observation: Video games primarily appeal to teenage boys [Slide:  boy playing video game vs. boy surfing porn]

\item for the most part, their girlfriends don't understand the continued fascination.  

\item Well, at least mine didn't.


\item Observation: Video games still appeal to 30-something married men [Slide: dad playing video games vs. dad surfing porn]

\item for the most part, their wives don't understand the continued fascination.
\item Well, at least mine doesn't.


\item Observation: Video games are at the center of a censorship debate [Slide: gamepolitics.com vs. The People vs. Larry Flynt]


\item Observation: Ebert used to review video games [Slide:  (Wired review 1994 vs. Deep Throat review 1973]

\item Close of the review:  ``This is a wonderful game.''

\item Ebert no longer reviews video games


\item Observation:  Video games want to be taken seriously, but it's usually a bit of a joke  [Slide:  Resident Evil 5 vs. Tell them Johnny Wadd is Here (1976)]

\item Bad acting is par for the course.

\item Observation:  Video games often pack their bad acting into cut scenes that interrupts the action [Slide:  GTA5 vs. Guardian 2008 porn opening scene]

\item many players fast-forward through the cut scenes to get to the good part


\item Observation:  Video games are obsessed with money shots. [Slide:  GTA San Andreas vs. Charlie White's ``Getting Lindsay Linton'' 2001]

\item Well, they can't get much worse than that, can they?  Moving on...  


\item Major advancements have been made recently:  game mechanics as a direct vehicle for artistic expression (goodbye cut scenes and linear narratives)[Slide:  advances] {\large \begin{itemize}
\item {\it Honorarium} (2008) by Ian Bogost
\item {\it Braid} (2008) by Jonathan Blow
\item {\it The Marriage} (2007) by Rod Humble
\item {\it Akrasia} (2008) by GAMBIT
\item {\it I wish I were the Moon} (2008) by Daniel Benmergui
\end{itemize} }

\item Artgame movement also detests the pursuit of addicting game designs, does not pander to the interests of teenage boys, and so on.  These games are meant to be played once or twice and thought about deeply.  Then you move on.

\item This approach seems to be helping games crawl out of the cultural gutter.  NPR thinks that these games are worth talking about now, for example.

\item But this approach also raises some other questions, and I'll close my talk with them.

\item Do we hate our own medium?

\item Are we throwing away what is unique about our medium as we strive for legitimacy?

\item For example, there's a fine line between addicting and compelling [Slide: old men playing chess]

\item Games are cultural artifacts that often demand a lifetime of study for full appreciation.

\item Sets the apart from other mediums

\item You only read a novel once or at most twice, but I've played Chess at least a dozen times, and I don't even play Chess.

\item Good games, at their heart, are {\it about} that kind of infinite intrigue.

\item Perhaps inherently obsession-inducing artifacts are simply not compatible with the notion of cultural legitimacy.

\item Perhaps they will never belong in that club [Return to Slide:  Line in the sand]

\item Perhaps this whole legitimacy quest is misguided in the the first place.

\item If that's the case, then what does the future of games hold?


\end{itemize}
}

\end{document}