\documentclass[12pt]{article}
\usepackage{fullpage,times}

%\pagestyle{empty}

\begin{document}

{\Huge

\begin{itemize}



\item Games have a major legitimacy problem.  Cultural ghetto.  [Slide:  line in the sand] {\large \begin{itemize}
\item {\it La Lecon} (1951) by Eugene Ionesco
\item {\it Lolita} (1955) by Vladimir Nabokov
\item {\it Taos, NM} (1947) by Henri Cartier-Bresson
\item {\it Blue Velvet} (1986) by David Lynch
\item {\it Guernica} (1937) by Picasso
\item {\it Model of the Monument to the Third International} (1920) by Vladimir Tatlin
\item {\it Sgt. Pepper's Lonely Hearts Club Band} (1967) by The Beatles
\item {\it Shadow of the Colossus} (2005) by Fumito Ueda
\item {\it The Legend of Zelda} (1986) by Shigeru Miyamoto
\item {\it Metal Gear Solid 2} (2001) by Hideo Kojima 
\end{itemize} }

\item How do we advance the medium?  For a while, aping film seemed to be promising. [Slide:  MGS in the limit]  

\item Dead end of aping film becoming more obvious (MGS4 had 6+ hours of cut scenes---perhaps more hours of cut scenes than gameplay).

\item Some mainstream games have dropped cut scenes entirely, and they're clearly the better for it. [Slide: BioShock]

\item Still stuck with the idea of an authored narrative.  Pockets of interactivity woven into an on-rails experience.  

\item All other mediums are each expressive in their own unique ways---none wannabe any other medium.  [Slide:  back to slide 1] Mainstream games wannabe films very badly. [Slide:  return to BioShock slide] ``Just like a movie, where you play the main character.''

\item Hurts---not helps---games' legitimacy problem.

\item How can games grow into a unique artistic medium?

\newpage

\item Major advancements have been made recently:  game mechanics as a direct vehicle for artistic expression (goodbye cut scenes and linear narratives; no more aping films) [Slide: example games] {\large \begin{itemize}
\item {\it Honorarium} (2008) by Ian Bogost
\item {\it Braid} (2008) by Jonathan Blow
\item {\it The Marriage} (2007) by Rod Humble
\item {\it Akrasia} (2008) by GAMBIT
\item {\it I wish I were the Moon} (2008) by Daniel Benmergui
\end{itemize} }

\item Each of these games could only exist as a game.  Artistic expression would not have been better served by a film.

\item So... we've done it!?  Games are art in a unique way!

\item But why are they still below the red line?  We're not there yet.

\item One guy has done more to advance the medium than anyone else. [Slide: Ebert]

\item My own ``Ebert Challenge'':  we need to show Ebert an artgame that will make him eat his hat.

\item What game would you show Ebert?  (List some)

\item Wise colleague said he'd show Ebert Go.  [Slide:  go board]

\item Profoundly deep gameplay from shockingly-simple rule-set.

\item Return to it throughout life and find something new (property of great art, no?)

\item So we know we need to make art through gameplay, and if it's great art, then that gameplay can never be exhausted...  Go-deep gameplay a requirement for art games?

\item But infinite replayability almost unheard of for video games.

\item Art games [Slide:  back to example art games]  Can return to many of these for a lifetime of {\it interpretation}, but not interesting gameplay.

\item Infinite replayability is actually quite common... for board games.  Chess, Checkers, Othello, plus hundreds of modern German designs (Settlers of Catan, El Grande, Tigris and Euphrates).

\item So what do these games have that video games don't?  Multiple players.

\item Up until about 30 years ago the phrase ``single player game'' was almost meaningless.  

\item ``Game'' by many definitions, requires multiple players.  Game theory doesn't even consider single-player games.


\item Very few examples of non-video single-player games [Slide: card solitaire] [Slide: peg solitaire]

\item By most defs., Peg Solitaire is a puzzle, not a game.  

\item Feels game-like because solution is too long to memorize (even after winning, playing again is challenging).

\item Card solitaire is a puzzle with randomized initial conditions (and incomplete information).

\item Randomization is fascinating (exploits a flaw in the human mind) [Slide: roulette]

\item Early arcade games used another trick to make games interesting for a single player:  reflex challenges [Slide:  Pac-Man]  Like juggling more and more balls.

\item So we have three techniques to make single-player games:  long, multi-step puzzles, randomization, and reflex challenges.  Combine these together, and you get.... [Slide:  Tetris]  Tetris, one of the most popular and addictive single-player video games of all time.

\item But there's something unsatisfying about that approach.  We certainly wouldn't think of showing Ebert {\it Tetris} as an artgame example (nor would we show him {\it Roulette}).

\item We aim higher.  Narrative is another ingredient that is added (cut scenes and linear stories that I bashed before).

\item Layer on mechanics? [Slide: Gravitation] (failure, for both art and non-art games)

\item Go is very simple but even simpler multiplayer games like the Prisoner's Dilemma, are deeply interesting.  No layers needed

\item Multiplayer is like fertile soil in which a system of mechanics can blossom into its full, emergent potential.  Rules are like DNA and the space of gameplay is the resulting organism.  Without multiplayer, even complex, layered systems of mechanics become stunted and can't blossom.


\item We might think the interest lies in the human factor:  matching wits against another mind (or against a close friend).

\newpage

\item What about AI?  AI vs. human *is* multiplayer.  Even AI vs. AI can allow deep gameplay to blossom.  Thus, human interest is not the key.

\item So what is the key?  The fundamentally interesting choice at the heart of all games:  ``If I do this and she does that and then I do this....'' on and on, down the rabbit hole that is each branch of the game tree. 

\item Where did the single player trend come from in the first place?  Tech limitations of early game systems (non-networked)

\item ``Multiplayer'' is the current big thing for video games [Slide: Halo]

\item But we've built multiplayer as a feature on top of our single-player game models (multiplayer reflex challenges).

\item we've come full circle, but missed the starting mark.

\item All this leads to some very interesting questions, and I'll close my talk with them.

\item Are we painting ourselves into the single-player corner?

\item Are we unwittingly cording ourselves off in the shallow end of the gameplay pool?

\item Are we missing what is in fact unique about our medium (the only medium that requires multiple participants to consume a work)?

\item What would a multiplayer artgame be like?

\item Is deep emergent gameplay compatible with authored artistic expression?

\item What happens when an emergent property of the gameplay is discordant with the artist's intentions?
 








\end{itemize}
}

\end{document}