\documentclass[12pt]{article}
\usepackage{fullpage,times}

%\pagestyle{empty}

\begin{document}

{\Huge

\begin{itemize}

\item Old purpose of religion:  explain physical mysteries [Slide:  mountains]

%\item Cherokee Buzzard creation myth

%{\normalsize
%The land was formed in the water from mud brought to the surface.  The land was still too wet so they sent the great buzzard from Galun'lati to prepare it for them. The buzzard flew down and by the time that he reached the Cherokee land he was so tired that his wings began to hit the ground. Wherever they hit the ground a mountain or valley formed.
%}

\item Aboriginal Dreaming [Slide:  Clifford Possum painting]

\item If it looks like a map, that's because it is

\item That purpose of religion replaced by science

\item Religion turned inward [Slide:  nuns]

\item But we still have physical mysteries

\item Still tell stories to explain them

%\item Don't normally think about this process as religious, but it bears all the same hallmarks.

%\item Mysteries surrounding human transformation of the earth

%\item Stories about human artifacts that we see

%\item Leads to a kind of humanistic spirituality

\item Example:  This is my grandfather, 1961 [Slide: Poticny Meeting]

\item Ohio highway director and two ohio senators.

\item tell story 

\item Cleveland to South Carolina [Slide:  whole 77 map]

\item Was going to bisect Fairlawn [Slide: close up of map]

\item I knew him, remember him

\item Real memories become vague

\item Larger-than-life stories grow in importance.

\item {\em Physical} traces grow in importance [Slide: sign]

\item Family to the park to see the sign

\item Or a drive past the house he supposedly built [Slide: 3136 Morewood]

\item How are these NOT pilgrimages?

\item Less man, more idea of a man

\item Real detail fade, replaced with imagined details

\item Remaining photos become icons [Slide: religious icons]

\item Holding onto his supposed utterances [Slide: Joe quotes]

\item And what is an idea of a man but a god?

\item Our connection with this idea becomes a spiritual, religious one

\item More dramatic the farther back in time

\item Never new my great grandfather

\item Driven past the house that he built

\item Every idea I have about him is imaginary

\item Same process for historical figures [Slide: Pollock's barn]

\item Visit to Pollock's barn is a pilgrimage

\item Taking this process to the limit, we get deep mysteries [Slide: Stonehenge]

\item We {\em imagine} the Druids, we really do

\item (Even though they didn't build the thing)

\item Summary of this humanistic religion:

\item {\em We become like gods to those who come after us.} [Slide]

%\item Flows both ways:  think of your connection to your great grand children

%\item Or to the people who might play your games after you're dead 

\item BREAK

\item How to foster this kind of religious practice through a game?

\item Meta game:  ``chain game'' [Slide: player chain]

\item A bit like a chain letter

\item Single person in the world plays the game at a time

\item Game passed on to next person in chain

\item Has properties of a religious practice:
\begin{itemize}
\item Chain could be a secret practice
\item Mystery about the nature of the game
\item Waiting to be chosen
\item Anticipation
\item You {\em might} get to have this experience sometime in your life
\end{itemize}

\item How does metagame foster predecessor/successor spiritual connection?

\item When you start a game, you assume that your start state IS the start state.
\item That's reality to you, your in-game foundation.

\item The game designer is God, in that case [Slide: player designer chain]

\item Understanding the game is in many ways like understanding that one god.

\item But what if your start state WASN'T actually a start state?

\item What if it was someone else's end state? [Slide: player state chain]

\item Example:  State of mario level is a given [Slide: mario state B]

\item Tunnel must be there for a reason?

\item But what if it {\em wasn't} a given? 

\item Instead, could be result of previous player's actions [Slide: mario state A]

\item Chain platformer is a bad idea

\item Challenges can be spent.

\item Previous player pretty much *ruined* that part of the level for the next player.

\item But what about a primary game with these properties:
\begin{itemize}
\item You have a goal to accomplish 
\item Modifying game world is a natural side-effect of pursuing that goal
\item Modifcations simply change the terrain that the next person will face
\item But terrain is not {\em ruined} by being modified
\end{itemize}

\item In a game with those properties, you might not realize how your modifications to the world will be perceived by the next player

\item Or you might realize and modify the world with intention.

%\item Leave traces behind for them to discover

%\item Lace meaning through your modifications

\item As next player, you will wonder how the world came to be in the state that it's in.

\item What was the previous player doing here that caused these gouges in the terrain?  

\item What is this road for?

\item You'll tell yourself stories to explain these things

\item I ran in circles for a while 

\item I tried to invent a new game with these properties

\item But my mind kept returning to an existing game.

\item A game where you modify the world as a side-effect of pursuing your goal.

\item A game where you leave traces behind and discover other people's traces.

\item And it also happens to be a game that has given me the most profoundly spiritual experiences that I've ever had with a game. [Slide:  minecraft]

\item My design is a minecraft mod

\item Or a minecraft meta-game

\item You receive it on a USB stick

\item THIS USB stick, the only one in the world.

\item This is the Ark.

\item Here are the rules: [Slide:  cannon law]

\item I was player one.

\item Someone in the audience gets to be player two.

\item According to law 7, is anyone interested?




\end{itemize}
}

\end{document}